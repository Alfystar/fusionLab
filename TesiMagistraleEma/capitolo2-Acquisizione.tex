\chapter{Architettura di Sistema}\label{systemDesign}

%\begin{minipage}{12cm}\textit{Se lo si desidera, utilizzare questo spazio per inserire un breve riassunto di ci\`o che verr\`a detto in questo capitolo. Inserire solo i punti salienti.}
%\end{minipage}

%\vspace*{1cm}

\section{Architettura ad alto livello}
Il progetto finale ha come obiettivo la realizzazione di un architettura di controllo per le bobine poloidali presenti nei reattori tokamak.

\begin{figure}[h]
	\centering
	\includegraphics[width=1\textwidth]{Architettura/SystemArchitetture.png}
	\caption[Schema finale dell'archiettettura di controllo]{Architettura di controllo}
\end{figure}

\noindent
Lo schema proposto realizza l'obiettivo è controllare una singola bobina, il progetto finale prevederà la ripetizione in serie del medesimo schema per il numero di bobine necessarie.\\

Dallo schema risulta evidente che tutti i componenti visti nel capitolo "\nameref{cap:1}" si relazionano con lo stesso \microControllore: l'\ArduinoUno.\\
Per riportare i dati fuori e ricevere il riferimento da inseguire nella $V_2$, è stata realizzato il \nameref{EMP}, essa è stata scritta in C++ affinché possa essere Cross-Platform.\\
Il suo compito specifico, in questo progetto, è di mettere in comunicazione l'\ArduinoUno con un nodo \MARTe installato su di una \Rasp.\\
Quest'ultimo nodo ha il compito di mettere in rete il feedback dell'esperimento, e comunicare all'\ArduinoUno eventuali cambio di riferimento. Questo ultimo tratto è realizzato mediante il protocollo \textbf{SDN}, che viaggia sopra Ethernet e dà garanzie Real-time.\\
Nella sua forma finale, il progetto prevede la riproduzione in serie di questo schema di controllo per arrivare a controllare tutte le bobine poloidali presenti in un tokamak.

\newpage

\section*{EMP - Libreria di Comunicazione Seriale\\Embedded Message Pack }\label{EMP}
\addcontentsline{toc}{section}{\protect\numberline{\thesection} EMP - Libreria di Comunicazione Seriale}

\begin{figure}[h]
	\centering
	\includegraphics[width=1\textwidth]{EMP/EMP-Logo-Background.png}
\end{figure}
\paragraph{EMP (Embedded Message Pack)} nasce con l’obiettivo di standardizzare un protocollo e creare una libreria C++ basata su classi Template, che permetta di automatizzare e standardizzare tutto il lavoro di programmazione necessario all’invio/ricezione di dei pacchetti dal formato Pre-Concordati tra 2 Device connessi Peer2Peer (Nessuna pretesa di network-ing).\\
Il raggiungimento dei suoi obiettivi, si sposa con la possibilità di supportare altre features interessanti:

\paragraph{Multiple-Package} Il protocollo di comunicazione che si è deciso di usare per EMP ha permesso di estendere il suo funzionamento e permettere il trasporto, attraverso lo stesso mezzo, di \textit{\textbf{pacchetti di tipologia e dimensione diversa}} all’interno della stessa libreria, evitando al contempo di inviare per ogni pacchetto più byte di quelli strettamente necessario. $\Rightarrow$ \textbf{Alta Efficienza}

\paragraph{Zero Tempo di negoziazione} Sempre grazie al protocollo di comunicazione, EMP è adatto ad un uso ‘Streaming’, questo perché non è necessario alcuna fase di sincronizzazione iniziale o durante la trasmissione in caso di perdita di dati, in aggiunta a ciò, EMP è in grado di scartare pacchetti errati in maniera trasparente all’utilizzatore. Tutto questo grazie al protocollo che \textbf{Auto-delimita i singoli pacchetti}. $\Rightarrow$ \textbf{Trasparenza Totale}

\paragraph{Responsabilità} Le uniche responsabilità a carico degli utilizzatori sono il riempimento dei pacchetti e la definizione degli stessi tra i 2 estremi della comunicazione.

\subsection*{Consistent Overhead Byte Stuffing (COBS)}
\addcontentsline{toc}{section}{\protect\numberline{\thesection} Protocollo - COBS}
Il protocollo di comunicazione che permette l’invio di \textbf{pacchetti diversi} e \textbf{senza fasi di negoziazione} alla base della libreria è \textbf{COBS}(\cite{COBS}).\\
Si tratta di un algoritmo per la codifica di byte, progettato per essere al tempo stesso efficiente e non ambiguo, che permette la definizione di \textit{data-pack frame} \textbf{Auto-delimiti} .

\begin{figure}[h]
	\centering
	\includegraphics[width=1\textwidth]{EMP/Cobs_encoding_with_example.png}
	\caption[Esempio di COBS]{Esempio di COBS}
\end{figure}

\subsubsection{title}


\subsection{Metodo di codifica}
\subsection{Struttura del codice}
\subsection{Benchmark}

\section{Online Sampling}
\subsection{Interconnessione Arduino $\Leftrightarrow$ Companion}
\subsection{Storage su file delle informazioni}


\section{Post Elaborazione con Matlab}
\subsection{Conversioni Dati}
\subsection{Creazione dei grafici e Filtraggio}
