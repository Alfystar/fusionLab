\chapter*{Appendice A\\ Arduino Code}\label{ArduinoCode}
\addcontentsline{toc}{chapter}{Appendice A - Codice Arduino}
\setcounter{chapter}{\thechapter + 1}

\section{Set-up Registri}
\subsection{Tic Timer}

\begin{lstlisting}[style=cppStyle,caption={Tic Timer},label=lst:ticTimer] 
	void periodicTask(int time) { 		// time in micro secondi
		// PWM pin Disable, motalita CTC(pt1)
		TCCR2A = (0x0 << COM2A0) | (0x0 << COM2B0) | (0x2 << WGM20);
		// CTC(pt2), Prescalere 256
		TCCR2B = (0 << WGM22) | (0x6 << CS20);                       
		// T_cklock * Twant / Prescaler = valore Registro
		OCR2A = (int)(16UL * time / 256);
		TIMSK2 = (1 << OCIE2A); // attivo solo l'interrupt di OC2A
	}
\end{lstlisting}
Questa funzione imposta il TIMER2 in modalità Fast PWM, ovvero che si resetta quando arriva al conteggio finale, e calcola il valore da mettere nel registro affinchè il conteggio sia il più vicino possibile a tempo desiderato

\subsection{Frequenza PWM}

\begin{lstlisting}[style=cppStyle,caption={Frequenza PWM},label=lst:pwmFreq] 
enum pwmFreq: char {
	hz30, hz120, hz490, hz4k, hz30k
};

void setMotFreq(pwmFreq freq) {
	// TCCR0B is for Timer 0
	#define myTimer TCCR0B
	switch (freq) {
		// set timer 3 divisor to  1024 for PWM frequency of    30.64 Hz
		case hz30:
			myTimer = (myTimer & B11111000) | B00000101;
		break;
		case hz120:
		// set timer 3 divisor to   256 for PWM frequency of   122.55 Hz
			myTimer = (myTimer & B11111000) | B00000100;
		break;
		case hz490:
		// set timer 3 divisor to    64 for PWM frequency of   490.20 Hz
			myTimer = (myTimer & B11111000) | B00000011;
		break;
		case hz4k:
		// set timer 3 divisor to     8 for PWM frequency of  3921.16 Hz
			myTimer = (myTimer & B11111000) | B00000010;
		break;
		case hz30k:
		// set timer 3 divisor to     1 for PWM frequency of 31372.55 Hz
			myTimer = (myTimer & B11111000) | B00000001;
		break;
		default:
			setMotFreq(hz4k);
		break;
	}
	#undef myTimer
}
\end{lstlisting}
Mediante questa funzione si modifica il valore del Prescaler per il TIMER 0, modificando la velocità di conteggio si ottiene un PWM con una periodo, e quindi frequenza, che varia.

\subsubsection{Offset Calculation}

\begin{lstlisting}[style=cppStyle,caption={Macro per calcolo Offset},label=lst:offsetCalc] 
#define offsetCalc(pin,n)            \
({                                   \
	long read = 0;                     \
	for (int i = 0; i < 1 << n; i++) { \
		read += analogRead(pin);         \
		delay(1);                        \
	}                                  \
	read = read >> 5;                  \
	read;                              \
})
\end{lstlisting}
\noindent
La macro misura con una distanza temporale di 1 ms $2^n$-volte, la misura del Pin, e successivamente ne fa la media eseguendo un revers shift, questa tecnica di divisione semplicemente minimizza il tempo di calcolo, poiché ogni revers shift equivale a una divisione per 2, e si ha l'effetto di dividere per $\frac{read}{2^n}$ in $O(n)$ con un operazione per il processore molto semplice.

\newpage
\section{Generatore di Segnale}

Per generare i segnali di controllo in Feed-Forward usati nel sistema, sono stati usati 2 diversi livelli di programmazione.\\
Un primo livello segnali di base, definiti su tutto $\mathbb{R}$, e usabili a piacere, e dei segnali compositi e periodici da mandare durante l'esperimento.
Tutti i segnali sono pensati per andare da -100\% <-> 100\%, è compito dell'attuazione
eliminare le deadzone e traslare il controllo al valore più opportuno

\subsection{Segnali Base}

\subsubsection{Rampa}
\begin{lstlisting}[style=cppStyle,caption={Rampa Saturata},label=lst:rampa] 
int ramp(uint64_t t, int vStart, uint64_t tStart, int vEnd, uint64_t tEnd) {
	// Saturazione
	if (t < tStart)
		return vStart;
	else if (t > tEnd)
		return vEnd;
	// Retta
	unsigned int dt = t - tStart;
	return vStart + int((vEnd - vStart) / float(tEnd - tStart) * dt);
}
\end{lstlisting}
La rampa è descritta come una retta nell'intervallo di interesse, saturata prima e dopo il tempo desiderato\\
$ RampaSat(t) =
	\left \{ \begin{array}{l c}
		v_{start} + \frac{v_{end}-v_{start}}{t_{end}-t_{start}} * (t - t_{start}) & \forall t \in [t_{start},t_{end}] \\
		v_{start}                                                                 & t<t_{start}                       \\
		v_{end}                                                                   & t>t_{start}
	\end{array}
	\right.
$

\newpage
\subsection{Segnali Composti}

\subsubsection{Onda Triangloare}
\begin{lstlisting}[style=cppStyle,caption={Onda Triangolare Periodica},label=lst:ondaTriangloare] 
int triangleSignal(uint64_t t, int msQuartPeriod) {
	static uint64_t startTic = 0;
	int dTic = t - startTic;
	int pwm = 0;
	if (dTic < ticConvert(msQuartPeriod))
		pwm = ramp(dTic, 0, 0, 100, ticConvert(msQuartPeriod));
	else if (dTic < (ticConvert(msQuartPeriod) * 3))
		pwm = ramp(dTic, 100, ticConvert(msQuartPeriod), -100, ticConvert(msQuartPeriod) * 3);
	else if (dTic < (ticConvert(msQuartPeriod) * 4))
		pwm = ramp(dTic, -100, ticConvert(msQuartPeriod) * 3, 0, ticConvert(msQuartPeriod) * 4);
	else {
		pwm = 0;
		startTic = t;
	}
	return pwm;
}
\end{lstlisting}

\subsubsection{Onda Trapezoidale (RapidShot)}
\begin{lstlisting}[style=cppStyle,caption={Onda Trapezoidale Periodica (RapidShot)},label=lst:ondaTrapezoidale] 
int rapidShot(uint64_t t) {
	static uint64_t startTic = 0;
	int pwmRapidShot;
	long dTic = t - startTic;
	if (dTic > t4) {
		startTic = t;
		pwmRapidShot = 0;
		dTic = t - startTic;
	}
	
	if (dTic <= t1) {
		pwmRapidShot = ramp(dTic, 0, 0, 100, t1);
	} else if (dTic <= t2) {
		pwmRapidShot = 100;
	} else if (dTic <= t3) {
		// falling ramp
		pwmRapidShot = ramp(dTic, 100, t2, 0, t3);
	} else if (dTic <= t4) {
		pwmRapidShot = 0;
	}	
	return pwmRapidShot;
}
\end{lstlisting}

\newpage

\section{Codici Controllore}

\subsubsection{Setup dell'esperimento}
\begin{lstlisting}[style=cppStyle,caption={Setup del Controllore},label=lst:controlSetup] 
void setup() {
	mpSerial.begin(2000000);
	
	memset(&pRead, 0, sizeof(pRead));
	memset(&pWrite, 0, sizeof(pWrite));
	
	// Motori
	setMotFreq(hz30k);
	mot = new DCdriver(enPwm, inA, inB);
	
	// Mean find
	pMean.V2_mean = offsetCalc(V2, 5);
	pMean.Isense_mean = offsetCalc(Isense, 5);
	pMean.dt = dtExperiment;
	
	// Ctrl Reference
	Ctrl.setNewRef(tic, 0);
	
	// ################# Start Experiment #################
	mpSerial.bufClear();
	// Start Delay
	memset(&pWrite, 0, sizeof(pWrite));
	pWrite.type = sampleType;
	delay(1000);
	periodicTask(pMean.dt);
	sei();
}
\end{lstlisting}

\subsubsection{Loop di Controllo}
\begin{lstlisting}[style=cppStyle,caption={Loop di Controllo},label=lst:controlLoop] 
volatile u32 oldTic = tic;
int readData;
int pwm;
void loop() {
	mpSerial.updateState();
	do {
		readData = mpSerial.getData_try(&pRead);
		if (readData >= 0) {
			serialExe(&pRead);
		}
	} while (tic == oldTic);
	
	pWrite.read.V2_read = analogRead(V2);
	pWrite.read.Isense_read = analogRead(Isense);
	pwm = mot->drive_motor(Ctrl.ctrlStep(oldTic, pWrite.read.V2_read - pMean.V2_mean));
	pWrite.read.pwm = pwm;
	pWrite.read.err = Ctrl.lastErr;
	
	oldTic++; // Suppose no over time
	mpSerial.packSend(&pWrite, sizeof(pWrite.type) + sizeof(pWrite.read));
}
\end{lstlisting}

\subsection{Classe Controllore}
\begin{lstlisting}[style=cppStyle,caption={Header Classe controllore C(s)},label=lst:controlClassH] 
// Common define
#define volt2adc(x) ((int)(((x) / (5.0 / (1023)) + 0.5)))
#define dtExperiment 500UL // us
#define ticConvert(ms) ((ms * 1000UL) / dtExperiment)
#define Ts (dtExperiment / 1000000.0) // second

class iiCTRL {
	// Saturation showdown experiment
	unsigned int ticSatCount = 0;
	// Adc Current Voltage reference
	int V2currRef = 0;
	
	float kp = 0.0;
	float k1 = 0.8;
	float k2 = 0.05;
	
	// State Space Cii(s)
	float xCii1 = 0, xCii2 = 0;
	// State Space Ci(s)
	float xCi1 = 0;
public:
	int lastCtrl;
	int lastErr;
	iiCTRL();
	iiCTRL(float k2, float k1, float kp);
	
	// State update
	void setNewRef(uint64_t ticSet, int v2AdcNewRef); // v2AdcNewRef senza offset
	int ctrlStep(uint64_t t, int v2Adc);              // v2Add gi senza offset
	
	// For switching logic
	void changeK(float k2, float k1, float kp);
	void changeK2(float k2);
	void changeK1(float k1);
	void changeKp(float kp);
	
private:
	void stateReset(int setOutput);
}
\end{lstlisting}
\noindent
La classe \verb|iiCTRL| implementa il controllore \ref{eq:controllerDesign} e usa le matrici di discretizzazione riportate nella tabella "\nameref{tab:discretizzazione}" per implementarle a tempo continuo.
\newpage

\begin{lstlisting}[style=cppStyle,caption={Source Classe controllore C(s)},label=lst:controlClassCpp] 
iiCTRL::iiCTRL() : iiCTRL(20000, 4000, 0.3) {}

iiCTRL::iiCTRL(float k2, float k1, float kp) {
	changeK(k2, k1, kp);
	setNewRef(0, 0);
}

void iiCTRL::setNewRef(uint64_t ticSet, int v2AdcNewRef) {
	V2currRef = v2AdcNewRef;
	ticSatCount = 0;
}

int iiCTRL::ctrlStep(uint64_t t, int v2Adc) {
	//Saturation Check
	if (ticSatCount >= ticConvert(200)){
		stateReset(0);
		return 0;
	}
	//Error Calc (ADC value), u_k per i sistema dinamico
	lastErr = (V2currRef - v2Adc);
	// Caso speciale: Riferimento a 0 => spengo tutto
	if (V2currRef == 0){
		stateReset(0);
		return 0;
	}
	// #### Space State Update ####
	// State update Cii(s)
	xCii1 = xCii1 + Ts * xCii2 + (sq(Ts) / 2.0) * k2 * lastErr;
	xCii2 = xCii2 + Ts * k2 * lastErr;
	// State update Ci(s)
	xCi1 = xCi1 + Ts * k1 * lastErr;
	float yCp = kp * lastErr;
	
	// #### Output calc ####
	lastCtrl = xCii1 + xCi1 + yCp; // y_k
	lastCtrl = constrain(lastCtrl, -1000, 1000); // Limitazione +-1000
	if (abs(lastCtrl) == 1000) {
		ticSatCount++;
		stateReset(lastCtrl);
	} else
	ticSatCount = 0;
	return lastCtrl;
}

void iiCTRL::stateReset(int setOutput) {
	// Riassegno tutto l'output su Cii(s)
	float rap = xCii1 / xCi1;
	xCii1 = setOutput;
	xCii2 = 0;
	xCi1 = 0;
}

void iiCTRL::changeK(float k2, float k1, float kp) {
	changeK2(k2);
	changeK1(k1);
	changeKp(kp);
}

void iiCTRL::changeK2(float k2) { this->k2 = k2; }
void iiCTRL::changeK1(float k1) { this->k1 = k1; }
void iiCTRL::changeKp(float kp) { this->kp = kp; }
\end{lstlisting}



\chapter*{Appendice B\\ EMP Code} \label{EMPCode}
\addcontentsline{toc}{chapter}{Appendice B - Codice EMP}
\setcounter{chapter}{\thechapter + 1}

\section{Pacchetti in EMP}
\subsection{Definizione di 2 Pacchetti}
Ecco di seguito un esempio di definizione asimmetrica per pacchetti da un Device (Linux), verso un altro (Arduino), e viceversa:\\
\begin{lstlisting}[style=cppStyle,caption={Esempio di definizione Pacchetto in EMP},label=lst:EMPpackDef] 
#include <stdint.h>	// To be sure for the footPrint in any platform
struct _packLinux2Ard {
	int16_t num;
	char buf[20];
} __attribute__((packed));
typedef struct _packLinux2Ard packLinux2Ard;

struct _packArd2Linux {
	int16_t num;
	char buf[10];
} __attribute__((packed));
typedef struct _packArd2Linux packArd2Linux;
\end{lstlisting}
\noindent
Come si può osservare, il sistema non necessita di alcun formato particolare per la definizione dei pacchetti, essi sono delle semplici strutture in C, con l'attributo \verb|__attribute__((packed))| per eliminare Padding e ottimizzare la memoria.
Come è possibile notare, le regole non vietano la definizione di 2 pacchetti diversi e asimmetrici.
\newpage

\subsection{Pacchetti Multipli}
Per definire un pacchetto di pacchetti, il metodo migliore è creare un \verb|enum| per numerare univocamente il tipo (nell'esempio \verb|uint8_t| ma può essere reso maggiore all'occorrenza), e una \verb|union| con tutti i tipi in fila.\\
Il pacchetto di interesse per \citefield{EMP}{title} è \verb|dataSend|, ed è responsabilità dell'utente della libreria riempire correttamente i campi.\\
\begin{lstlisting}[style=cppStyle,caption={Definizione Pacchetti Multipli},label=lst:EMPmultiplePack] 
//DataType1 ,DataType2, DataType3 typedef above;
enum packTypeSend : uint8_t{
	DataType1code,
	DataType2code,
	DataType3code,
};
typedef union{
	DataType1 t1;
	DataType2 t2;
	DataType3 t3;
}  __attribute__((packed)) dataSendUnion;

typedef struct{
	packTypeSend type;
	dataSendUnion pack;
}  __attribute__((packed)) dataSend;
\end{lstlisting}
\noindent
Per ottimizzare l'invio al minimo strettamente necessario si può quindi usare la variante nel metodo di \verb|packSend(...)|, in cui viene specificata la dimensione:

\begin{lstlisting}[style=cppStyle,caption={Definizione Pacchetti Multipli},label=lst:EMPsendOptimize] 
//...
dataSend.type = DataType2code;
dataSend.pack.t2 = /*{data}*/ // fill correctly dataSend
//...    
MP->packSend(&data, sizeof(packTypeSend) + sizeof(DataType2));
//...
\end{lstlisting}


\chapter*{Appendice C\\ Matlab Post Elaborazione}\label{MatlabCode}
\addcontentsline{toc}{chapter}{Appendice C - Matlab Post Elaboration}
\setcounter{chapter}{\thechapter + 1}

\section{Importazione dei Dati}
\subsection{Parsing Tabelle} \label{subsec:tabParsing}
\begin{lstlisting}[style=matlabStyle,caption={Parsing delle tabelle},label=lst:tabParsing] 
function [Table,Info] = importExperiment(filename)
	% #####################
	% #### Mean Import ####
	% #####################
	opts = delimitedTextImportOptions("NumVariables", 4);
	% Specify range and delimiter
	opts.DataLines = [2, 2];
	opts.Delimiter = "\t";
	% Specify column names and types
	opts.VariableNames = ["V2_mean", "Isense_mean", "dt", "V2Ref_set"];
	opts.SelectedVariableNames = ["V2_mean", "Isense_mean", "dt", "V2Ref_set"];
	opts.VariableTypes = ["double", "double", "double", "double" ];
	% Specify file level properties
	opts.ExtraColumnsRule = "ignore";
	opts.EmptyLineRule = "read";	
	% Import the data
	Info = readtable(filename, opts);
	Info.dt = Info.dt * 1E-6; % dt shuld be in second
	[~ , name, ~] = fileparts(filename);
	Info.Properties.Description = name;	
	% #####################
	% #### Data Import ####
	% #####################
	% Set up the Import Options and import the data
	opts = delimitedTextImportOptions("NumVariables", 3);
	dataLines = [4, Inf];
	% Specify range and delimiter
	opts.DataLines = dataLines;
	opts.Delimiter = "\t";
	% Specify column names and types
	opts.VariableNames = ["PWM", "V2_read", "Isense_read"];
	opts.VariableTypes = ["double", "double", "double"];
	% Specify file level properties
	opts.ExtraColumnsRule = "ignore";
	opts.EmptyLineRule = "read";
	% Import the data
	Table = readtable(filename, opts);
	Table = tableConvert(Table,Info);
end
\end{lstlisting}
\newpage
\subsection{Conversione dei Dati}
Il parsing delle tabelle si appoggia alla conversione e filtraggio dei dati qui riportato:
\begin{lstlisting}[style=matlabStyle,caption={Parsing delle tabelle},label=lst:dataFilter]
function tCon = tableConvert(table,Info)
	% Utility Factor
	vScale = 5/(2^10-1);        % ADC sensitivity
	IScale = vScale / 0.020;    % Isensitivity = 20mv/A
	V2Mean = Info.V2_mean;
	IsenseMean = Info.Isense_mean;
	dt = Info.dt;
	vRef = Info.V2Ref_set;
	% New Table
	tCon = table;
	tCon.PWMDead = pwm2duty(tCon.PWM,60,210);
	tCon.PWM=tCon.PWM/255;
	tCon.V2_read = (table.V2_read - V2Mean) * vScale;
	tCon.Isense_read = (table.Isense_read - IsenseMean) * IScale;
	tCon.error = (tCon.V2_read - vRef * vScale);
	
	d = designfilt('lowpassfir', ...
	'PassbandFrequency',0.01,'StopbandFrequency',0.4, ...
	'PassbandRipple',1,'StopbandAttenuation',60, ...
	'DesignMethod','equiripple');
	tCon.V2_readFilter = filtfilt(d,tCon.V2_read);
end
\end{lstlisting}
Usando come funzioni accessorie:
\begin{lstlisting}[style=matlabStyle,caption={Parsing delle tabelle},label=lst:deadzoneRemove] 
function u = pwm2duty(pwmList, downDead, upDead)
	u = zeros(length(pwmList),1);
	for i = 1:length(pwmList)
		pwm = pwmList(i);
		if(abs(pwm)>=upDead)
			u(i)=sign(pwm);
			continue
		end
		if(abs(pwm)<=downDead)
			u(i) = 0;
			continue
		end
		if(pwm>0)
			u(i) = map(pwm,downDead,upDead,0,1);
			continue
		else
			u(i) = -map(-pwm,downDead,upDead,0,1);
			continue
		end
	end
end

function u = map (var, varMin, varMax, outMin , outMax)
	u = ((var-varMin)/(varMax-varMin));
	u = (u * outMax) + ((1-u)*outMin);
end
\end{lstlisting}


\section{Stima Parametri}

\subsection{Stima parametri automatica}
\begin{lstlisting}[style=matlabStyle,caption={Stima corrente Primario automatica},label=lst:autoEst] 
% ##############################################
% ########### PWM-I2 Auto Estimation ###########
% ##############################################
% DAT = iddata(Y,U,Ts) 
PrimaryFrameData = iddata(experiment.Isense_read, experiment.PWMDead, Info.dt);

% ########### PWM-I1 Estimation ###########
% DAT = iddata(Y,U,Ts) 
InPriRiseFrameData = iddata(experiment.Isense_read(1:1200), experiment.PWMDead(1:1200), Info.dt);

% SYS = tfest(DATA, NP, NZ, IODELAY)
tfExtPWM_I1 = tfest(InPriRiseFrameData, 1, 0, NaN)
% lsim(SYS,U,T)
yExtPWM_I1Fit = lsim(tfExtPWM_I1,experiment.PWMDead(1:1200),t(1:1200));

% ########### Primary - Secondary Rise Estimation ###########
% DAT = iddata(Y,U,Ts) 
PriSecRiseFrameData = iddata(experiment.V2_read(1:1200), yExtPWM_I1Fit, Info.dt);
tfExtI1_V2 = tfest(PriSecRiseFrameData, 1, 1, NaN)

% ########### Simulation ###########
yExtPWM_I1 = lsim(tfExtPWM_I1,experiment.PWMDead,t);
yExtI1_V2 = lsim(tfExtI1_V2,yExtPWM_I1,t);

figure(1)
plotTable(experiment,Info,[0.2,1.2],[-inf,inf], [yExtPWM_I1, yExtI1_V2], {'yExtPWM-I1', 'yExtI1-V2'}, 'autoExt');
\end{lstlisting}

\subsection{Stima parametri manuale}
\begin{lstlisting}[style=matlabStyle,caption={Stima parametri manuale},label=lst:manualEst] 
% ##############################################
% ########## PWM-I2 Manual Estimation ##########
% ##############################################
fPWM_I1 = tf(8.4,[0.022 1]);
yfPWM_I1 = lsim(fPWM_I1,experiment.PWMDead,t);

fI1_V2 = tf([0.0085 0],[1/2800 1]);
yfI1_V2 = lsim(fI1_V2,yfPWM_I1,t);

figure(2)
plotTable(experiment,Info,[0.2,1.2],[-inf,inf], [yfPWM_I1, yfI1_V2], {'yfPWM-I1','yfI1_V2'},'manualExt');
\end{lstlisting}

\newpage

\subsection{Plot dell'esperimento}
\begin{lstlisting}[style=matlabStyle,caption={Plot dell'esperimento},label=lst:plotTable] 
function plotTable(table, Info,xLim, yLim, ySim, ySimLegend , nameSave)
	V2Mean = Info.V2_mean;
	IsenseMean = Info.Isense_mean;
	dt = Info.dt;
	vRef = Info.V2Ref_set;
	
	name = Info.Properties.Description;
	[len , ~ ] = size(table);
	
	t = [0:len-1]' * dt;
	
	clf
	grid on
	hold on
	plot(t,table.PWMDead,'LineWidth',2)
	plot(t,table.V2_readFilter,'LineWidth',2)
	plot(t,table.Isense_readFilter,'LineWidth',2)
	for y = ySim
		plot(t,y,'LineWidth',1);
	end
	legNorm = {'PWM%-DeadFilter','V_{secondary}','I_{primary}'};
	
	legend(cat(2,legNorm,ySimLegend));
	
	title("Experiment: "+name+ "   SampleTime: " + mat2str(dt * 1000000) + "us")
	xlabel("time [s]")
	xlim(xLim);
	ylim(yLim);
	
	% Figure Export and Save
	savePath = "./img/";
	[~,~] = mkdir(savePath);
	set(gcf, 'PaperUnits', 'normalized')
	set(gcf, 'PaperPosition', [0 0 1 1])
	set(gcf,'PaperOrientation','landscape');
	saveas(gcf,savePath + name + '-' + nameSave + '.png')
	saveas(gcf,savePath + name + '-' + nameSave + '.pdf')
end
\end{lstlisting}