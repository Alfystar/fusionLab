\chapter*{Appendice A\\ Arduino Code}\label{ArduinoCode}
\addcontentsline{toc}{chapter}{Appendice A - Codice Arduino}

\section{Setup Registri}

\section{Generatore di Segnale}

Per generare i segnali di controllo in Feed-Forward usati nel sistema, sono stati usati 2 diversi livelli di programmazione.\\
Un primo livello segnali di base, definiti su tutto $\mathbb{R}$, e usabili a piacere, e dei segnali compositi e periodici da mandare durante l'esperimento.
Tutti i segnali sono pensati per andare da -100\% <-> 100\%, è compito dell'attuazione
eliminare le deadzone e traslare il controllo al valore più opportuno

\subsection{Segnali Base}

\subsubsection{Rampa}
\begin{lstlisting}[style=cppStyle,caption={Rampa Saturata},label=lst:rampa] 
int ramp(uint64_t t, int vStart, uint64_t tStart, int vEnd, uint64_t tEnd) {
	// Saturazione
	if (t < tStart)
		return vStart;
	else if (t > tEnd)
		return vEnd;
	// Retta
	unsigned int dt = t - tStart;
	return vStart + int((vEnd - vStart) / float(tEnd - tStart) * dt);
}
\end{lstlisting}
La rampa è descritta come una retta nell'intervallo di interesse, saturata prima e dopo il tempo desiderato\\
$ RampaSat(t) = 
\left \{ \begin{array}{l c}
	v_{start} + \frac{v_{end}-v_{start}}{t_{end}-t_{start}} * (t - t_{start}) &\forall t \in [t_{start},t_{end}] \\
	v_{start} & t<t_{start}\\
	v_{end}& t>t_{start}
\end{array}
\right.
$

\newpage
\subsection{Segnali Composti}

\subsubsection{Onda Triangloare}
\begin{lstlisting}[style=cppStyle,caption={Onda Triangolare Periodica},label=lst:ondaTriangloare] 
int triangleSignal(uint64_t t, int msQuartPeriod) {
	static uint64_t startTic = 0;
	int dTic = t - startTic;
	int pwm = 0;
	if (dTic < ticConvert(msQuartPeriod))
		pwm = ramp(dTic, 0, 0, 100, ticConvert(msQuartPeriod));
	else if (dTic < (ticConvert(msQuartPeriod) * 3))
		pwm = ramp(dTic, 100, ticConvert(msQuartPeriod), -100, ticConvert(msQuartPeriod) * 3);
	else if (dTic < (ticConvert(msQuartPeriod) * 4))
		pwm = ramp(dTic, -100, ticConvert(msQuartPeriod) * 3, 0, ticConvert(msQuartPeriod) * 4);
	else {
		pwm = 0;
		startTic = t;
	}
	return pwm;
}
\end{lstlisting}

\subsubsection{Onda Trapezoidale}
\begin{lstlisting}[style=cppStyle,caption={Onda Trapezoidale Periodica},label=lst:ondaTrapezoidale] 
int rapidShot(uint64_t t) {
	static uint64_t startTic = 0;
	int pwmRapidShot;
	long dTic = t - startTic;
	if (dTic > t4) {
		startTic = t;
		pwmRapidShot = 0;
		dTic = t - startTic;
	}
	
	if (dTic <= t1) {
		pwmRapidShot = ramp(dTic, 0, 0, 100, t1);
	} else if (dTic <= t2) {
		pwmRapidShot = 100;
	} else if (dTic <= t3) {
		// falling ramp
		pwmRapidShot = ramp(dTic, 100, t2, 0, t3);
	} else if (dTic <= t4) {
		pwmRapidShot = 0;
	}	
	return pwmRapidShot;
}
\end{lstlisting}


\chapter*{Appendice B\\ EMP Code}\label{EMPCode}
\addcontentsline{toc}{chapter}{Appendice B - Codice EMP}

\chapter*{Appendice C\\ Matlab Post Elaborazione}\label{MatlabCode}
\addcontentsline{toc}{chapter}{Appendice C - Matlab Post Elaboration}