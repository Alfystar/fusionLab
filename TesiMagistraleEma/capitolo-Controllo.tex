\chapter{Sviluppo Controllo reale}\label{cap:controlDevelop}

\begin{minipage}{12cm}\textit{
		In questo capitolo vedremo come è implementato nella realtà il controllore descritto nel capitolo \nameref{cap:controlModel} all'interno del \microControllore, tareremo i suoi coefficienti e mostreremo come si comporta in un esperimento reale con tutti i problemi ad esso connessi (\nonLinearita del \cite*{IBT-2}, errori di quantizzazione, discretizzazione del controllo, ecc...)
	}
\end{minipage}

\vspace*{1cm}
\noindent
Essendo un \microControllore un computer non eccessivamente potente, è necessario realizzare la funzione di trasferimento del il controllo (equazione \ref{eq:controllerDesign}) mediante un sistema a tempo discreto.\\
Per semplificarci l'implementazione, invece di creare un unico sistema del 2° ordine, si è optato per sommare tra loro le tre funzioni di trasferimento semplici che compongono il controllore, così da semplificare il debugging e permettere un più semplice controllo sullo stato dei componenti, utile per saturare gli integratori una volta che l'uscita ha superato la soglia di attuabilità (\textit{Saturazione}).\\
In fine, essendo l'obiettivo di controllo realizzato da una rampa, dopo un certo tempo in saturazione, per evitare lo spreco di energia da parte della batteria che alimenta il sistema, il codice disattiva il controllo, resetta tutti gli stati e mette il riferimento a 0. Questo stato non è permanente ma persiste fino al raggiungimento di un nuovo input di controllo mediante \cite*{EMP}.

\newpage
\section{Discretizzazione Zero-Order Hold (Z.O.H.)}
Un oggetto del tipo \textit{\textbf{Zero-Order Hold} (Z.O.H.)} altro non è che un convertitore Digitale$ \rightarrow $Analogico che permette di interfacciare dei segnali \textbf{Tempo Discreto} $\mathbb{T} =  \mathbb{Z} $ che evolvono $\forall \Delta t $ con sistemi dinamici \textbf{Tempo Continuo} $\mathbb{T} =  \mathbb{R} $.

\begin{figure}[H]
	\centering
	\caption[Discretizzazione Zero-Order Hold  $ H_d(z) $ del sistema Tempo continuo $ H(s) $]{Discretizzazione ZOH $ H_d(z) $ per il sistema Tempo continuo $ H(s) $}
	\includegraphics[width=1\textwidth]{Controllo/zoh-apply.png}
\end{figure}
\noindent
Questa interconnessione "\textit{congela}" l'ultimo segnale discreto $ u(k T_s) $ ricevuto e lo ripropone come un segnale costante in ingresso al sistema continuo che evolve in maniera indipendente.\\
L'uscita $ y(k T_s) $ è la discretizzazione di $ y(t) $, campionata ogni $ T_s $ \footnote{$ T_s $ = Sampling Time}.
\begin{figure}[H]
	\centering
	\caption[Effetto sui segnali discretizzati con il metodo Zero-Order Hold]{Segnali discretizzati con il metodo Zero-Order Hold}
	\includegraphics[width=1\textwidth]{Controllo/SegnaliDiscretiContinui.png}
\end{figure}
\noindent
Il sistema $ H(s) $ durante gli intervalli $ T_s $  evolve avendo una costante in ingresso, e al successivo istante di campionamento l'uscita raggiunta viene campionata e mantenuta per ul successivo $ T_s $, e questo procedimento all'infinito.
\newpage
\noindent
Ora che abbiamo descritto qualitativamente cosa avviene usando una discretizzazione ZOH, vediamo ora come diventa $ H_d(s) $ nello spazio di stato in termini matematici. Il procedimento di discretizzazione necessita di passare attraverso lo spazio di stato dei 3 sistemi dinamici che compongono il controllore \ref{eq:controllerDesign} e usando le formule del professore \cite{Discretizzazione}, si ottengono i risultati seguenti risultati:
\begin{table}[H]
	\centering
	\caption[Funzioni di trasferimento nello spazio di Stato, da Tempo Continuo a Tempo Discreto]{Funzioni di trasferimento nello spazio di Stato, da $ \mathbb{T} = \mathbb{R} \rightarrow \mathbb{T} = \mathbb{Z} $}\label{tab:discretizzazione}
	{\Large
		\begin{tabular}[t]{||c||c||c||}
			\hline
			                                                                          &                                        &                         \\[-3mm]
			$ C_{I^2}(s) = \frac{K_2}{s^2}$                                           & $ C_I(s) = \frac{K_1}{s}$              & $ C_p(s) = K_p $        \\[2mm]
			                                                                          &                                        &                         \\[-3mm]
			{\normalsize $ \left\{\begin{matrix}
					\dot{x} = & \begin{pmatrix}
						0 & 1 \\
						0 & 0
					\end{pmatrix} x & + & \begin{pmatrix}
						0 \\
						K_2
					\end{pmatrix} u \\
					          &                                                               \\[-1mm]
					y       = & \begin{pmatrix}
						1 & 0
					\end{pmatrix} x
				\end{matrix}\right. $
			}                                                                         &
			$ \left\{\begin{matrix}
					\dot{x} = & x & + K_1 \cdot u \\
					y       = & x &
				\end{matrix}\right.$

			                                                                          &
			$\left\{\begin{matrix}
					y = K_p \cdot u
				\end{matrix}\right. $                                                                                                   \\[9mm]
			\hline\hline
			                                                                          &                                        &                         \\[-3mm]
			{\normalsize $ \left\{\begin{matrix}
					x^+ = & {\small \begin{pmatrix}
								1 & T_s \\
								0 & 1
							\end{pmatrix}} \cdot x_k & + & K_2 {\small \begin{pmatrix}
						T_s^2/2 \\
						T_s
					\end{pmatrix}} u_k \\
					      &                                                                                                \\[-1mm]
					y_k = & \begin{pmatrix}
						1 & 0
					\end{pmatrix} \cdot x_k
				\end{matrix}\right. $
			}                                                                         & 
		{\normalsize $ \left\{\begin{matrix}
							x^+ = & x_k & + K_1 T_s \cdot u_k \\
							y_k = & x_k
						\end{matrix}\right.$

			}                                                                         &
			$\left\{\begin{matrix}
					y_k        = K_p \cdot u_k
				\end{matrix}\right. $                                                                                                  \\[9mm]
			                                                                          &                                        &                         \\[-3mm]
$ C_{I^2}(z)|_{T_s} = K_2 \cdot \frac{T_s}{2} \cdot \frac{z+1}{(z -1)^2}$ & $ C_I(z)|_{T_s} = \frac{K_1 T_s}{z-1}$ & $ C_p(z)|_{T_s} = K_p $ \\[2mm]

			\hline
		\end{tabular}
	}%\Large
\end{table}\vspace{-3mm}
\noindent
Si può notare che in tutti e 3 i sistemi si è scelta una realizzazione della funzione di trasferimento nello Spazio di Stato tempo continuo in \textit{Forma Compagna di Osservabilità} (\cite{FormeCanoniche}).\\
Questa scelta non è stata casuale e ha lo scopo di semplificare in futuro la codifica per un sistema di controllo \textbf{Switching nei Coefficienti}, volto a migliorarne ulteriormente le prestazioni, senza dover scalare lo stato in base al cambio dei coefficienti.\\
Questa comoda proprietà è dovuta alla struttura della matrice $ C $, la quale, per variazioni \textbf{istantanee} dei coefficienti $ K_2,K_1,K_p$, non propaga gli effetti sull'uscita $ y(t) $, l'effetto della variazione arriverà in un secondo momento grazie all'integrazione dello stato, che ovviamente evolverà diversamente da prima a causa delle variazioni dei coefficienti. Per avere un idea qualitativa degli effetti dei coefficienti sul sistema complessivo, rifarsi alla sezione "\nameref{sec:designControlloreConclusioni}".\\
I calcoli della trasformata Zeta, sono presenti solo per completezza e sono stati ottenuti usando la classica formula:\vspace{-5mm}
\begin{center}
{\Large 		$ G(z) = C_d \left(z I - A_d\right)^{-1} B_d + D_d $}
\end{center}
\noindent
Per i nostri scopi noi siamo interessati alle matrici $ A_d,B_d,C_d,D_d $ della conversione, così da implementare esattamente la discretizzazione di ordine Zero del sistema dinamico.\\


\section{Codifica del controllore}
La codifica del controllore \ref{eq:controllerDesign} avviene attraverso le matrici $ A_d,B_d,C_d,D_d $ riportate in tabella \ref{tab:discretizzazione}, con queste matrici è stato costruito la classe di controllo \verb|iiCTRL| riportata in appendice nel codice del controllore (\nameref{lst:controlClassH} e \nameref{lst:controlClassCpp}).\\
La classe viene chiamata all'interno del \nameref{lst:controlLoop} ogni istante di campionamento, e permette la modifica a richiesta del riferimento da inseguire.\\
Al suo interno ha già la logica di \textit{Safe-Shutdown} da attivare quando il controllo ha raggiunto la saturazione per un tempo superiore ai 200ms, risulta infatti rischioso tenere connesso il trasformatore, poiché le correnti che scorrono sono dell'ordine dei 10A, e in ogni caso raggiungere gli obiettivi di controllo necessita di un rampa di corrente sul primario, che non si può ottenere se si è già saturato.\\
Questo contatore viene ripristinato a 0 all'arrivo di un nuovo riferimento da inseguire, causando di fatto il ripristino dell'esperimento.\\

\section{Esperimenti}

