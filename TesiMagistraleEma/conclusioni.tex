\chapter{Conclusioni e sviluppi futuri}
\section*{Conclusioni}
In conclusione, con questo prototipo si è dimostrato e realizzato un controllore \textit{PID-style} a doppio Polo nell'origine, è un design di controllore perfetto da usare per governare la corrente di una bobina in un impianto Tokamak.\\
La realizzazione di questo prototipo però, come per ogni progetto pratico, ha dovuto confrontarsi con vari problemi di \nonLinearita, problemi implementativi, limiti di attuazione e campionamento, etc...\\
La risoluzione di questi problemi ha portato a compromessi e semplificazioni volte a catturare gli aspetti principali dell'esperimento, trascurando quelli secondari.\\
Il controllo così creato, oltre a funzionare bene nella teoria, risulta robusto alle variazioni dal modello lineare presenti nella realtà ma trascurate durante la modellizzazione del sistema, e ciò rende un simile design di controllo general-purple per impianti tokamak, poiché i coefficienti trovati nel corso di questa tesi permettono l'ottimizzazione per questo impianto in particolare, ma porterebbero a convergenza un qualunque impianto Tokamak avente la stessa struttura nella dinamica.\\

\section*{Sviluppi Futuri}
Anche se per i fini della tesi il progetto termina qui, il progetto in se ancora avanza, e tra gli sviluppi futuri abbiamo:
\begin{itemize}
	\item L'intenzione di portare un controllo \textbf{Switching} nella legge di Up-date, variando i coefficienti del controllore \ref{eq:controllerDesign}\\
	      In tal senso il controllore implementato nel \microControllore, ottenuto dal modello nello spazio di stato in Forma Compagna di Osservatore (\cite{FormeCanoniche}) già rende il sistema pronto per questo tipo di modifica a livello di codice, è necessario tuttavia studiare e simulare per quali soglie lo switch dei parametri può incrementare le performance, prima di poter implementare una simile tecnica.
	\item Attualmente è in cantiere la realizzazione all'interno di \MARTe della libreria \cite*{EMP} per permettere l'integrazione di questo firmware con l'ecosistema \MARTe.
	\item Risulta di interesse eseguire un upgrade del \microControllore dal'\ArduinoUno a una scheda più performante, e che permetta comunicazioni, campionamenti e PWM a velocità superiori, permetterebbe di superare l'attuale soglia di $ 2Khz $ di controllo, e realizzare dei controlli molto più performanti con tempi di campionamento molto inferiori, e quindi ritardi di fase ancora più azzerati.
	\item In fine, per raggiungere performance superiori è necessario avere un motore superiore, in questo caso il ponte-H, trovarne uno con una dinamica più lineare e che possa funzionare a frequenze maggiori di PWM permetterebbe di avere un controllo sul Primario molto più fine e potente, portando a migliorare le prestazioni in maniera sensibile e tangibile.

\end{itemize}




