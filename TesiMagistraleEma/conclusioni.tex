\chapter{Conclusioni e sviluppi futuri}
\section*{Conclusioni}
In conclusione, con questo prototipo si è realizzato un controllore \textit{PID-style} a doppio Polo nell'origine, il quale si è dimostrato esser un design di controllore adatto ad un generico impianto Tokamak.\\
La realizzazione di questo prototipo però, come per ogni progetto pratico, ha dovuto confrontarsi con vari problemi:\\
\nonLinearita nel prototipo reale, problemi implementativi, limiti di attuazione e campionamento, etc...\\
La risoluzione di questi problemi ha portato a compromessi e semplificazioni volte a catturare gli aspetti principali dell'esperimento, trascurando quelli secondari.\\
Il controllo così creato, oltre a funzionare bene nella teoria, risulta robusto alle variazioni dal modello lineare, e ciò rende un simile design di controllo \textit{general-purple} per impianti Tokamak, poiché i coefficienti trovati nel corso di questa tesi permettono l'ottimizzazione per questo impianto in particolare, ma porterebbero a convergenza un qualunque impianto Tokamak avente la stessa struttura nella dinamica.\\

\section*{Sviluppi Futuri}
Anche se per i fini della tesi il lavoro termina qui, il progetto in se ancora avanza, e tra gli sviluppi futuri abbiamo:
\begin{itemize}
	\item L'intenzione di portare un controllo \textbf{Switching} nella legge di Update, variando i coefficienti del controllore \ref{eq:controllerDesign}\\
	      In tal senso il controllore implementato nel \microControllore, ottenuto dal modello nello spazio di stato in Forma Compagna di Osservatore (\cite{FormeCanoniche}) già rende il sistema pronto per questo tipo di modifica a livello di codice, è necessario tuttavia studiare e simulare per quali soglie lo switch dei parametri può incrementare le performance, prima di poter implementare una simile tecnica.
	\item Attualmente è in cantiere la realizzazione all'interno di \MARTe della libreria \cite*{EMP} per permettere l'integrazione di questo firmware con l'ecosistema \MARTe.
	\item Risulta di interesse eseguire un upgrade del \microControllore dal'\ArduinoUno a una scheda più performante, e che permetta comunicazioni, campionamenti e PWM a frequenze superiori, superando così l'attuale soglia di $ 2Khz $ nel controllo.
	\item In fine, per raggiungere performance superiori è necessario avere un motore superiore, in questo caso il Ponte-H, trovarne uno con una dinamica più lineare e che possa funzionare a frequenze maggiori di PWM permetterebbe di avere un controllo sul Primario molto più fine e potente, portando a migliorare le prestazioni in maniera sensibile e tangibile.
\end{itemize}




