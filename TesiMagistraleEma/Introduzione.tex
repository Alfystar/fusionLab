\chapter*{Ringraziamenti}
\addcontentsline{toc}{chapter}{Ringraziamenti}
\vspace{-8mm}
Questa tesi è stata resa possibile dal contributo nella mia vita di tante persone, che giorno per giorno mi hanno sempre dato il loro sostegno, a voi dedico questa mia Tesi.\\
Un ringraziamento speciale alla mia famiglia, in particolare a mia \textbf{Madre} e mio \textbf{Padre}: è grazie al vostro sostegno e incoraggiamento se oggi sono riuscito a raggiungere questo traguardo.\\
La forza di arrivare qui, oggi, però non è dovuta solo a loro, devo per forza ringraziare dell'affetto e il sostegno speciale da parte dei miei cari amici, che ogni giorno hanno condiviso con me gioie, sacrifici e successi, senza voltarmi mai le spalle, mi hanno dato la forza di arrivare a questo prezioso traguardo. \textbf{Filippo}, \textbf{Gabriele}, \textbf{Marta}, grazie di TUTTO.\\
Un pensiero in particolare vola verso la mia dolce \textbf{\textit{Nicoleta}}, è sicuramente grazie all'affetto e le attenzioni che mi hai donato che sono riuscito a tenere dritto il timone ed arrivare qui oggi.
Per terminare voglio ringraziare tutti i professori che negli ultimi 18 anni hanno guidato il mio cammino, loro che hanno sempre creduto in me e nelle mie capacità meritano tutta la mia gratitudine.\\
Un ringraziamento speciale va però alla mia professoressa e mentore \textbf{Beniamina Rauch} che fu la prima a vedere il mio potenziale e coltivarlo, so che ancora vegli su di me, non troverò mai le parole per ringraziarti abbastanza.\\
Oltre a lei ringrazio il mio relatore \textbf{Daniele Carnevale} che in questi anni universitari, da quando mi ha conosciuto ad oggi, ha sempre creduto in me e mi ha permesso di fare esperienze che mai avevo immaginato.\\ \\
Un sentito grazie a tutti voi e buona lettura.


\chapter*{Introduzione}
\addcontentsline{toc}{chapter}{Introduzione}
Il presente lavoro di tesi si colloca nel contesto delle tecniche di controllo per impianti Tokamak come
FTU (Frascati Tokamak Upgrade), attualmente in smaltimento, Proto-Sphera o ITER (International Thermonuclear Experimental Reactor).\\
La tesi è stata realizzata in collaborazione con il centro ricerche ENEA (Ente Nazionale per l’Energia e l’Ambiente) di Frascati ed ha come obiettivo complessivo la replica in miniatura di un intero impianto Tokamak.
\\L'aspetto di cui mi sono occupato nello sviluppo della tesi è stata la creazione del prototipo di controllo delle bobine magnetiche che permettono il controllo e confinamento magnetico del plasma all'interno dell'impianto del Tokamak.

\section*{Fusione Termonucleare, di cosa si tratta}
\addcontentsline{toc}{section}{Fusione Termonucleare, di cosa si tratta}
In fisica, per \textbf{Fusione Termonucleare}, si intende il processo mediante il quale i nuclei di due o più atomi vengono compressi tanto da far prevalere l’interazione forte sulla repulsione coulombiana, causando l'unione tra gli atomi e andando a formare così un nucleo la cui massa totale risulta minore della massa dei reagenti originali. La perdita di massa ha come conseguenza la liberazione di un’elevata quantità di energia, la quale conferisce al processo caratteristiche fortemente esotermiche.\\
La massa mancante viene trasformata in energia in accordo con l’equazione di Einstein:
\begin{center}
	$E = (m_r - m)c^2$
\end{center}
Dove $ m_r $ è la massa dei reagenti e $ m $ è la massa risultante.\\
Il processo di \textbf{Fusione Nucleare} avviene naturalmente all'interno delle stelle, e trasforma l'\textit{idrogeno} di cui sono composte in \textit{elio}.\\
L'energia nucleare prodotta a seguito di questa reazione è elevatissima, e per tale motivo risulta di forte interesse per la civiltà umana sviluppare tecniche sofisticate che ne permettano la raccolta in maniera controllata.\\
Tra i vantaggi di questa tecnologia troviamo motivi sia economici che ambientali, i principali possono essere sintetizzati in:
\begin{description}
	\item [Combustibile semi-inesauribile]:\\ 
	Il combustibile (idrogeno e/o deuterio) è praticamente inesauribile ed è a disposizione di tutte le nazioni che abbiano uno sbocco sul mare. Il deuterio può essere estratto dall'acqua, anche se con costi energetici non indifferenti; per fare un esempio, un ditale pieno di deuterio equivale a 20 tonnellate di carbone in termini di energia. Un lago di medie dimensioni contiene deuterio sufficiente a rifornire una nazione di energia per secoli utilizzando la fusione nucleare (ovviamente supponendo di sfruttarlo tutto).
	\item [Rischi di Contaminazione grave \underline{nulli}]:\\
	 Nessuna possibilità di incidenti come quelli di Černobyl' o di Three Mile Island in quanto il reattore non contiene sostanze radioattive come l'uranio o le scorie di fissione. In oltre, possibili incidenti, come fughe di trizio o perdite di liquido refrigerante, avrebbero un impatto ambientale e radiativo molto più contenuto e temporaneo.
	\item [Assenza di Inquinanti Ambientali]:\\
	 Nessun prodotto chimico da combustione (anidride carbonica ad esempio) come residuo immesso nell'atmosfera e quasi nessun contributo al riscaldamento del pianeta.
	\item [Nessuno prodotto derivato adatto a fini Bellici]:\\
	 Impossibilità di utilizzo dei reattori per la produzione di materiale per scopi bellici o terroristici
	\item [Livelli di radiottività residua lievi]:\\
	 Basso livello di radioattività residua e produzione di sostanze con breve vita media (tempo in cui la radioattività si riduce rapidamente).
\end{description}
\noindent
Tutti questi vantaggi rendono lo sviluppo di Centrali termonucleari a \textbf{Fusione Nucleare} di grande interesse e tecnologie chiave per preservare l'ambiente del nostro pianeta.\\
In questa tesi faremo riferimento a impianti sperimentali di tipo Tokamak, che però, va specificato essere solo uno dei design attualmente in sviluppo nelle ricerche sulla \textbf{Fusione Nucleare Controllata.}
\newpage

\section*{Struttura di un Tokamak}
\addcontentsline{toc}{section}{Struttura di un Tokamak}
Un \textbf{Tokamak} (acronimo russo per "camera toroidale magnetica") è una macchina di forma toroidale (a ciambella) in cui un gas (solitamente idrogeno) viene portato nello stato di plasma e mantenuto coeso e lontano dalle pareti interne grazie ad un campo magnetico creato da elettromagneti esterni alla camera.
\begin{figure}[H]
	\centering
	\caption[Sezione di un Tokamak]{Sezione di un Tokamak}
	\includegraphics[width=0.65\textwidth]{Introduzione/K-DEMO_device_core_design_features.jpg}
\end{figure}

\noindent
La forma del Tokamak a \textit{toro} (ciambella) è studiata per permettere alle particelle del plasma di muoversi all'interno del campo magnetico, creato all'esterno delle pareti (\textit{Esterno del Vessel}) in un moto circolare.\\
Questo movimento avviene poiché le particelle del plasma sono per definizione cariche, e in quanto tali, se immerse in un campo magnetico esse tendono a muoversi seguendo una traiettoria elicoidale (detta anche \textit{moto di ciclotrone}) attorno alle linee del campo magnetico, che in questo caso sono chiuse e contenute all'interno della sezione del Tokamak (\textit{Interno del Vessel}).\\
L'uso di una confinazione magnetica per questo plasma è dovuta all'impossibilità, per qualunque materiale, di resistere alle enormi temperature raggiunte dal plasma durante la fusione in un contatto diretto con esse.
\begin{figure}[H]
	\centering
	\caption[Geometria di un Toro]{Geometria di un Toro}
	\includegraphics[width=0.65\textwidth]{Introduzione/toro-big.png}
\end{figure}\vspace{-8mm}
\noindent
La scelta della forma toroidale deriva dalla fisica, in particolare dall'equazione di Larmor, che definisce il raggio di Larmor:\\
\begin{vwcol}[widths={0.3,0.7}, sep=8mm, rule=1px]
	\begin{empheq}[box=\mathCalc]{equation*} \label{eq:Larmor}
		{\displaystyle \,\rho ={\frac {mv_{\perp }}{Ze\,B}}}
	\end{empheq}
	\newpage % con wcol, le colonne sono "pagine"
	\begin{spacing}{1.25}
		{\footnotesize
			$ {\displaystyle v_{\perp }} $ è la velocità della particella perpendicolare al campo magnetico.\\
			$ {\displaystyle m} $ è la sua massa.\\
			$ {\displaystyle B} $ è l'intensità del campo magnetico.\\
			$ {\displaystyle Ze} $ è la carica del portatore.
		}
	\end{spacing}
\end{vwcol}
\noindent
Come conseguenza di questa equazione abbiamo che il plasma, contenuto all'interno del Vessel, non si può espandere più di $ {\displaystyle \rho } $ dalla circonferenza $ K $, che corrisponde alla linea di campo magnetico generato dall'\textit{Esterno del Vessel} per contenere il plasma. Questa geometria di controllo forma un \textbf{Toro} per il contenimento del plasma, e attorno a questo Toro virtuale vi è il Tokamak con la sua forma toroidale reale.\\
In sintesi un Tokamak è un impianto progettato per creare questo confinamento magnetico in maniera efficace e sicura, e ha la forma più consona per realizzare questo obiettivo.\\
Oltre al mero contenimento però, un impianto Tokamak si prefigge l'obiettivo di compattare il plasma su se stesso (aumentando la forza del campo magnetico $ B $), così da avvicinare di più gli atomi tra loro (l'aumento di $ B $ causa una riduzione di $ \rho $).\\
Questo aumento di pressione, unito alle alte energie immesse nel Tokamak (sotto forma di temperatura, le quali sono dell'ordine di decine di volte la temperatura sulla superficie del sole) aumentano le probabilità di scontro tra gli atomi, e quindi inizio di un processo di fusione tra essi. 

\section*{Obiettivi della Tesi}
\addcontentsline{toc}{section}{Obiettivi della Tesi}
Il progetto complessivo portato avanti dall'ENEA prevede la prototipazione di tutti i livelli di un sistema Tokamak, in vista del prossimo Tokamak che verrà costruito nei laboratori dell'ENEA.
\begin{figure}[H]
	\centering
	\caption[Architettura Complessiva Progetto ENEA]{Architettura Complessiva Progetto ENEA}
	\includegraphics[width=1\textwidth]{Introduzione/ArchitetturaComplessiva.png}
\end{figure}
\noindent
La mia parte di progetto ({\color{red}Alfano})  consiste nello sviluppo del prototipo della scheda di controllo per le bobine di un impianto Tokamak in grado di ricevere i riferimenti desiderati per la corrente di plasma (argomento trattato con dettaglio nella tesi) attraverso la rete di interconnessione tra i dispositivi di controllo dell'impianto Tokamak, basata sul Framework di \MARTe, e attuare un controllo opportuno sulla corrente del primario per inseguire il riferimento con errore nullo.\\
In fine il dispositivo deve essere in grado di comunicare in real-time l'attuale stato del sistema al resto della rete, per permettere una diagnostica in tempo reale dell'andamento dell'impianto.\\