\chapter*{Ringraziamenti} 
\addcontentsline{toc}{chapter}{Ringraziamenti}
Corpo dei ringraziamenti
%Questa tesi è stata resa possibile dal contributo nella mia vita di tante persone, che giorno per giorno mi hanno sempre dato il loro sostegno, a voi dedico questa mia Tesi.\\
%Un ringraziamento speciale alla mia famiglia, in particolare a mia \textbf{Madre} e mio \textbf{Padre}: è grazie al vostro sostegno e incoraggiamento se oggi sono riuscito a raggiungere questo traguardo.\\
%La forza di arrivare qui, oggi, però non è dovuta solo a loro, devo per forza ringraziare dell'affetto e il sostegno speciale da parte dei miei cari amici, che ogni giorno hanno condiviso con me gioie, sacrifici e successi, senza voltarmi mai le spalle, mi hanno dato la forza di arrivare a questo prezioso traguardo.
%\textbf{Filippo}, \textbf{Gabriele}, \textbf{Marta}, grazie di TUTTO.\\
%Un pensiero in particolare vola verso la mia dolce \textbf{\textit{Nicoleta}}, è sicuramente grazie all'affetto e le attenzioni che mi hai donato che sono riuscito a tenere dritto il timone ed arrivare qui oggi.
%Per terminare voglio ringraziare tutti i professori che negli ultimi 18 anni hanno guidato il mio cammino, loro che hanno sempre creduto in me e nelle mie capacità. Un ringraziamento più speciale va però alla mia professoressa e mentore \textbf{Beniamina Rauch} che fu la prima a vedere il mio potenziale e coltivarlo.\\
%Oltre a lei ringrazio il mio relatore \textbf{Daniele Carnevale} che in questi anni universitari, da quando mi ha conosciuto, ha sempre creduto in me e mi ha permesso di fare esperienze che mai avevo immaginato.\\ \\
%Un sentito grazie a tutti voi.\\
%%TODO: Aggiungere firma con tavoletta frafica
%
%\chapter*{Introduzione}
%\addcontentsline{toc}{chapter}{Introduzione}
%
%Il capitolo introduttivo \`e generalmente lungo tre pagine (almeno due). 
%Una buona introduzione pu\`o essere preparata secondo il seguente schema caratterizzante tre blocchi consecutivi: 
%\begin{enumerate}
%\item {\em Introduzione generale all'ambito in cui si colloca la tesi} (pi\`u o meno partendo da ``caro amico"). Ad esempio: ``La robotica nasce dall'esigenza di sostituire l'uomo in quei lavori che... " eccetera.
%
%\item {\em Collocazione della tesi nell'ambito generale sopra descritto}. Ad esempio: ``Questo lavoro di tesi si colloca nel contesto dell'automazione domestica. In particolare, con riferimento a quanto sopra accennato, l'esigenza di .... ''.
%
%\item {\em Descrizione schematica della struttura della relazione} (un paragrafo o poco pi\`u). Ad esempio: ``La tesi \`e strutturata come segue: nel Capitolo~\ref{cap:primo} viene discussa una ..., 

%\end{enumerate}

