\documentclass[a4paper, 15pt]{report}
%\usepackage[latin1]{inputenc}
\usepackage[utf8]{inputenc}
\usepackage[italian]{babel}
\usepackage[T1]{fontenc}
\usepackage{graphicx}
\usepackage{float}
\usepackage[centertags]{amsmath}
\usepackage{amsfonts}
\usepackage{amssymb}
\usepackage{amsthm}
\usepackage{newlfont}
\usepackage{fancyhdr}
\usepackage{tesisty}

%-------------------------------
% DEFINIZIONE DEGLI ENVIRONMENT
%-------------------------------

\newtheorem{obs}{Osservazione}[section]
\newenvironment{oss}
    {\begin{obs}\begin{normalfont}}
    {\hfill $\square \!\!\!\!\checkmark$ \end{normalfont}\end{obs}}

\newtheorem{pro}{Problema}[chapter]
\newenvironment{prob}
    {\begin{pro}\begin{normalfont}}
    {\hfill $\spadesuit$ \end{normalfont}\end{pro}}

\newtheorem{teor}{Teorema}[section]
\newenvironment{teorema}
    {\begin{teor}\textit }
    {\hfill  \end{teor}}

\newtheorem{defn}{Definizione}[section]
\newenvironment{de}
    {\begin{defn}\begin{normalfont}}
    {\hfill $\clubsuit$ \end{normalfont}\end{defn}}


% CUSTOM PACKAGE && ENVIROMENT SET
% Custom Color Package
\usepackage{xcolor}
\definecolor{dkgreen}{rgb}{.133,.545,.133}
\definecolor{mauve}{rgb}{0.224, 0.176, 0.255}
\definecolor{NavyBlue}{HTML}{000080}
\definecolor{Azure}{HTML}{007FFF}

%% Link Package
\usepackage{hyperref}
\hypersetup{
	colorlinks=true,
	linkcolor=Azure, %black, %blue,
	filecolor=black, %magenta,      
	urlcolor=cyan,
	citecolor=cyan,
	pdftitle={Tesi Magistrale Alfano},
	pdfpagemode=FullScreen,
}

%% Graphics Package
\usepackage{graphicx}
\graphicspath{{imgs/}}

%% Bibliografia Package
\usepackage[
style=numeric-comp,
useprefix,
style=reading,
backend=biber]{biblatex}
\addbibresource{BibliografiaTesi.bib}
\usepackage{csquotes}

%\usepackage{natbib}
%\bibliographystyle{unsrtnat}
%\bibliography{BibliografiaTesi}

%% Enumeration personalization Package
%\usepackage{enumitem}
%\setlist[enumerate]{itemsep=-4mm}

%% Coding Package
\usepackage{listings}

% define general style
\lstdefinestyle{defaultStyle}
{
	basicstyle=\footnotesize,
	tabsize=2,
	captionpos=b,
	frame=lines,
	breaklines=true,
	keepspaces=true
}
% define C++ style
\lstdefinestyle{cppStyle}
{
	style=defaultStyle,
	% language related
	language=C++,
	keywordstyle=\color{blue},
	commentstyle=\color{dkgreen},
	stringstyle=\color{mauve},
	showstringspaces=false,
	%otherkeywords={\#include}, % do not uncomment!
	% numbering
	numbers=left,
	numberstyle=\tiny
}

%-----------------------------
% CONFIGURAZIONE DELLA PAGINA
%-----------------------------

\hfuzz2pt % Don't bother to report over-full boxes if over-edge is < 2pt

\fancypagestyle{plain}{
\fancyhead{}\renewcommand{\headrulewidth}{0pt} } \pagestyle{fancy}
\renewcommand{\chaptermark}[1]{\markboth{\small CAP. \thechapter \textit{ #1}} {} }
\renewcommand{\sectionmark}[1]{\markright{\small  \thesection \textit{ #1}} {} }
\voffset=-20pt    % distanza tra il limite superiore del foglio e l'intestazione
\headsep=40pt     % distanza  l'intestazione ed il testo del corpo
\hoffset=0 pt     % misura equivalente al margine sinistro
\textheight=620pt % altezza del corpo del testo
\textwidth=435pt  % larghezza del corpo del testo
\footskip=40pt    % distanza tra il testo del corpo ed il pie' di pagina
\fancyhead{}      % cancella qualsiasi impostazione per l'intestazione
\fancyfoot{}      % cancella qualsiasi impostazione per il pie' di pagina
\headwidth=435pt  % larghezza del'intestazione e del pie' di pagina
\fancyhead[R]{\rightmark} \fancyfoot[L]{\leftmark}
\fancyfoot[R]{\thepage}
\renewcommand{\headrulewidth}{0.3pt}   % spessore della linea dell'intestazione
\renewcommand{\footrulewidth}{0.3pt}   % spessore della linea del pi�di pagina

\numberwithin{equation}{section}
\renewcommand{\theequation}{\thesection.\arabic{equation}}




%--------------------------
% MODIFICARE DA QUI IN POI
%--------------------------

\begin{document}

\dedicate{{
		Dedico questa tesi ai miei cari nonni.\\
		Grazie per aver sempre creduto in me.
}}
\corso{DELL'AUTOMAZIONE}
\titoloTesi{
	Sviluppo di algoritmi di controllo delle correnti nelle bobine poloidali di macchine per la fusione Tokamak, con riguardo al design	sistemico per la cooperazione tra sistemi embedded per l'attuazione, misurazione e centrali di controllo.
} \anno{2020/2021}
\relatore{Daniele Carnevale}
\autore{Emanuele Alfano}
\correlatore{Marco Passeri}

\baselineskip=25pt

\intestazione

%------------------------------------------------
% INTRODUZIONE E RINGRAZIAMENTI (NON MODIFICARE)
%------------------------------------------------

\fancypagestyle{plain}{
	\fancyhead{}\renewcommand{\headrulewidth}{0pt} } \pagestyle{fancy}
\renewcommand{\chaptermark}[1]{\markboth{\small Cap. \thechapter \textit{ #1}} {} }
\renewcommand{\sectionmark}[1]{\markright{\small  \S \thesection \textit{ #1}} {} }
\voffset=-20pt                         % distanza tra il limite superiore del foglio e l'intestazione
\headsep=40pt                          % distanza  l'intestazione ed il testo del corpo
\hoffset=0pt                           % misura equivalente al margine sinistro
\textheight=620pt                      % altezza del corpo del testo
\textwidth=435pt                       % larghezza del corpo del testo
\footskip=40pt                         % distanza tra il testo del corpo ed il pie' di pagina
\fancyhead{}                           % cancella qualsiasi impostazione per l'intestazione
\fancyfoot{}                           % cancella qualsiasi impostazione per il pie' di pagina
\headwidth=435pt                       % larghezza del'intestazione e del pie' di pagina
\fancyhead[R]{\rightmark} \fancyfoot[L]{\leftmark}
\fancyfoot[R]{\thepage}
\renewcommand{\headrulewidth}{0.3pt}   % spessore della linea dell'intestazione
\renewcommand{\footrulewidth}{0.3pt}   % spessore della linea del pi�di pagina

\pagenumbering{Roman}
%Indice, senza link colorati
{
	\hypersetup{hidelinks}
	\tableofcontents
}
\newpage

\pagenumbering{arabic}

\fancyhead[R]{Introduzione} \fancyfoot[L]{Introduzione}
\fancyfoot[R]{\thepage}

\chapter*{Ringraziamenti}
\addcontentsline{toc}{chapter}{Ringraziamenti}
\vspace{-8mm}
Questa tesi è stata resa possibile dal contributo nella mia vita di tante persone, che giorno per giorno mi hanno sempre dato il loro sostegno, a voi dedico questa mia Tesi.\\
Un ringraziamento speciale alla mia famiglia, in particolare a mia \textbf{Madre} e mio \textbf{Padre}: è grazie al vostro sostegno e incoraggiamento se oggi sono riuscito a raggiungere questo traguardo.\\
La forza di arrivare qui, oggi, però non è dovuta solo a loro, devo per forza ringraziare dell'affetto e il sostegno speciale da parte dei miei cari amici, che ogni giorno hanno condiviso con me gioie, sacrifici e successi, senza voltarmi mai le spalle, mi hanno dato la forza di arrivare a questo prezioso traguardo. \textbf{Filippo}, \textbf{Gabriele}, \textbf{Marta}, grazie di TUTTO.\\
Un pensiero in particolare vola verso la mia dolce \textbf{\textit{Nicoleta}}, è sicuramente grazie all'affetto e le attenzioni che mi hai donato che sono riuscito a tenere dritto il timone ed arrivare qui oggi.
Per terminare voglio ringraziare tutti i professori che negli ultimi 18 anni hanno guidato il mio cammino, loro che hanno sempre creduto in me e nelle mie capacità meritano tutta la mia gratitudine.\\
Un ringraziamento speciale va però alla mia professoressa e mentore \textbf{Beniamina Rauch} che fu la prima a vedere il mio potenziale e coltivarlo, so che ancora vegli su di me, non troverò mai le parole per ringraziarti abbastanza.\\
Oltre a lei ringrazio il mio relatore \textbf{Daniele Carnevale} che in questi anni universitari, da quando mi ha conosciuto ad oggi, ha sempre creduto in me e mi ha permesso di fare esperienze che mai avevo immaginato.\\ \\
Un sentito grazie a tutti voi e buona lettura.


\chapter*{Introduzione}
\addcontentsline{toc}{chapter}{Introduzione}
La tesi si colloca nell'ambito delle tecniche di controllo per impianti Tokamak, come FTU (Frascati Tokamak Upgrade), Proto-Sphera ed ITER (International Thermonuclear Experimental Reactor).\\
Essa è stata realizzata in collaborazione con il centro ricerche \textbf{ENEA} (Ente Nazionale per l’Energia e l’Ambiente) di Frascati e coordinato dal gruppo \textbf{CODAS} (COntrol and Data Acquisition System).\\
L'obiettivo finale del CODAS è di gestire il sistema di controllo ed acquisizione di un intero impianto Tokamak.\\
In questa tesi viene discussa la creazione e lo sviluppo del prototipo dell'architettura e le leggi di controllo per l'asservimento della corrente nelle bobine magnetiche, generando così il confinamento magnetico del Plasma necessario alla reazione di fusione nucleare, che avviene all'interno del Tokamak.
\newpage

\section*{Struttura di un Tokamak}
\addcontentsline{toc}{section}{Struttura di un Tokamak}
Un \textbf{Tokamak}\footnote{Tokamak è l'acronimo russo per "camera toroidale magnetica"} è una macchina di forma toroidale (a ciambella) in cui un gas (solitamente idrogeno) viene portato nello stato di Plasma e mantenuto coeso e lontano dalle pareti interne grazie ad un campo magnetico creato da elettromagneti esterni alla camera.\\
Scopo di un Tokamak è generare in maniera controllata reazioni di \textbf{fusione nucleare}.\\
\begin{figure}[H]
	\centering
	\caption[Sezione di un Tokamak]{Sezione di un Tokamak}
	\includegraphics[width=0.65\textwidth]{Introduzione/K-DEMO_device_core_design_features.jpg}
\end{figure}

\noindent
La forma del Tokamak a \textit{toro} è studiata per permettere alle particelle del Plasma di muoversi all'interno del campo magnetico, creato all'esterno delle pareti (\textit{Esterno del Vessel}) in un moto circolare.\\
Questo movimento avviene poiché le particelle del Plasma sono per definizione cariche, e in quanto tali, se immerse in un campo magnetico esse tendono a muoversi seguendo una traiettoria elicoidale (detta anche \textit{moto di ciclotrone}) attorno alle linee del campo magnetico, che in questo caso sono chiuse e contenute all'interno della sezione del Tokamak (\textit{Interno del Vessel}).\\
L'uso di una confinazione magnetica di questo Plasma è dovuta all'impossibilità, per qualunque materiale, di resistere alle enormi temperature raggiunte dal Plasma durante la fusione in caso di un contatto diretto.\\

\noindent
La forma scelta per i Tokamak a Toro ha motivazioni fisiche, che discendono principalmente dall'equazione di Larmor. L'equazione descrive la distanza massima oltre il quale una particella carica non può allontanarsi da una linea di campo magnetico, definendo un limite superiore per lo spazio di contenimento delle particelle; questa distanza è detto appunto \textit{raggio di Larmor}:\\
\begin{vwcol}[widths={0.3,0.7}, sep=8mm, rule=1px]
	\begin{empheq}[box=\mathCalc]{equation*} \label{eq:Larmor}
		{\displaystyle \,\rho ={\frac {mv_{\perp }}{Ze\,B}}}
	\end{empheq}
	\newpage % con wcol, le colonne sono "pagine"
	\begin{spacing}{1.25}
		{\footnotesize
			$ {\displaystyle v_{\perp }} $ è la velocità della particella perpendicolare al campo magnetico.\\
			$ {\displaystyle m} $ è la sua massa.\\
			$ {\displaystyle B} $ è l'intensità del campo magnetico.\\
			$ {\displaystyle Ze} $ è la carica del portatore.
		}
	\end{spacing}
\end{vwcol}
\noindent
L'equazione vale anche in presenza di un campo magnetico curvo, se questa curva è chiusa si ottiene che la particella è confinata in uno spazio di volume finito ma rimane comunque libera ruotare all'infinito, accumulando così energia.\\
\'E stata scelta la geometria del \textbf{Toro} per rendere questa linea di campo una circonferenza esatta, la quale semplifica enormemente delle complesse equazioni fluido-dinamiche applicate al Plasma.\vspace{-5mm}
\begin{figure}[H]
	\centering
	\caption[Geometria di un Toro]{Geometria di un Toro}
	\includegraphics[width=0.65\textwidth]{Introduzione/toro-big.png}
\end{figure}\vspace{-8mm}
\noindent
In sintesi un Tokamak è un impianto progettato per creare questo confinamento magnetico in maniera efficace e sicura, e ha la forma più consona per realizzare questo processo.\\
Oltre al mero contenimento però, il secondo obiettivo è quello di compattare il Plasma su se stesso (aumentando la forza del campo magnetico $ B $), così da avvicinare di più gli atomi tra loro (l'aumento di $ B $ causa una riduzione di $ \rho $).\\
Questo aumento di pressione, unito alle alte energie immesse nel Tokamak (la cui temperatura è \textbf{di decine di volte} superiore a quella della superficie del sole) aumentano le probabilità di scontro tra gli atomi, e quindi inizio di un processo di fusione tra essi. 

\newpage

\section*{Cos'è la Fusione Termonucleare?}
\addcontentsline{toc}{section}{Cos'è la Fusione Termonucleare?}
In fisica, per \textbf{Fusione Termonucleare}, si intende il processo mediante il quale i nuclei di due o più atomi vengono compressi tanto da far prevalere l’interazione forte sulla repulsione coulombiana, causando l'unione tra gli atomi e andando a formare così un nucleo la cui massa totale risulta minore della massa dei reagenti originali. La perdita di massa ha come conseguenza la liberazione di un’elevata quantità di energia, la quale conferisce al processo caratteristiche fortemente esotermiche.\\
La massa mancante viene trasformata in energia in accordo con l’equazione di Einstein:
\begin{center}
	$E = (m_r - m)c^2$
\end{center}
Dove $ m_r $ è la massa dei reagenti e $ m $ è la massa risultante.\\
Il processo di \textbf{fusione nucleare} avviene naturalmente all'interno delle stelle e trasforma l'\textit{idrogeno} (di cui sono composte) in \textit{elio}.\\
L'energia nucleare prodotta a seguito di questa reazione è elevatissima e per tale motivo risulta di forte interesse per la civiltà umana sviluppare tecniche sofisticate che ne permettano la raccolta in maniera controllata.\\
Tra i vantaggi di questa tecnologia troviamo sia motivi economici che ambientali e i principali possono essere sintetizzati in:
\begin{description}
	\item [Combustibile semi-inesauribile:]\phantom{.}\\
	      Il combustibile (idrogeno e/o deuterio) è praticamente inesauribile ed è a disposizione di tutte le nazioni che abbiano uno sbocco sul mare. Il deuterio può essere estratto dall'acqua, anche se con costi energetici non indifferenti;\\
	      per fare un esempio, un ditale pieno di deuterio equivale a 20 tonnellate di carbone in termini di energia.\\
	      Un lago di medie dimensioni contiene deuterio sufficiente a rifornire una nazione di energia per secoli utilizzando la fusione nucleare (ovviamente supponendo di sfruttarlo tutto).
	\item [Rischi di Contaminazione grave sono \underline{nulli}:]\phantom{.}\\
	      Nessuna possibilità di incidenti come quelli di Černobyl' o di Three Mile Island in quanto il reattore non contiene sostanze radioattive come l'uranio o le scorie di fissione.\\
	      In oltre, possibili incidenti, come fughe di trizio o perdite di liquido refrigerante, avrebbero un impatto ambientale e radiativo molto più contenuto e temporaneo.
	\item [Assenza di inquinanti ambientali:]\phantom{.}\\
	      Il processo di fusione, durante il quale gli atomi di idrogeno vengono fusi per creare elio, non rilascia nessun elemento inquinante da combustione (ad esempio anidride carbonica) e per tanto nessun residuo può essere immesso nell'atmosfera impedendo il riscaldamento del pianeta tramite l'effetto serra.
	\item [Nessuno prodotto derivato è adatto a fini Bellici:]\phantom{.}\\
	      La mancanza di materiale radioattivo residuo, impedisce l'utilizzo dei reattori per la produzione di materiale per scopi bellici o terroristici, come proiettili radioattivi o materiale fissile per bombe nucleari.
	\item [Livelli lievi di radioattività residua:]\phantom{.}\\
	      Il processo di fusione può rendere radioattivi solo i materiali che compongono il Tokamak e i gas usati per il Plasma, ma entrambi gli elementi possiedono un tempo di decadimento alquanto breve, il ché implica che i rifiuti radioattivi hanno una scarsa durata nel tempo, in oltre le radiazioni prodotte non sono a energie elevatissime, semplificandone il contenimento attraverso contenitori schermati più economici e sicuri.
\end{description}
\noindent
Questi e molti altri motivi rendono lo sviluppo di metodi per catturare l'energia da \textbf{Fusione Nucleare} con attraverso processi controllati di grande interesse per tutta la civiltà umana, essa potrebbe infatti essere la chiave per preservare l'ambiente del nostro pianeta e fornire al temo stesso l'energia di cui l'umanità necessita.\\
In questa tesi faremo riferimento ad una classe di impianti sperimentali per la \textbf{Fusione Nucleare Controllata}, detti "\textbf{Impianti Tokamak}". \'E doveroso specificare che essi sono solo uno dei tanti design su cui attualmente si fa ricerca e sviluppo nel mondo, ma tra le varie proposte essi sembrano essere i più promettenti per aprire le porte a questa rivoluzione, poiché il principio di base è la replica delle condizioni che generano la fusione all'interno in una stella.\\
Tolte le difficoltà tecniche per le quali infatti è ancora una tecnologia in sviluppo, abbiamo la sicurezza sperimentale della fattibilità del processo, la nostra stessa civiltà non esisterebbe se il Sole non producesse questa forma di energia al suo interno.
\newpage


\section*{Obiettivi della Tesi}
\addcontentsline{toc}{section}{Obiettivi della Tesi}
Il progetto complessivo portato avanti dall'ENEA prevede la realizzazione di prototipi a più livelli di un sistema Tokamak.\vspace{-4mm}
\begin{figure}[H]
	\centering
	\caption[Architettura Complessiva Progetto ENEA]{Architettura Complessiva Progetto ENEA}
	\includegraphics[width=\textwidth]{Introduzione/ArchitetturaComplessiva.png}
\end{figure}\vspace{-8mm}
\noindent
La mia parte di progetto ({\color{red}Alfano}) consiste nello sviluppare e realizzare una una scheda embedded.\\
Essa dovrà pilotare la corrente nelle bobine dell'impianto Tokamak ed essere in grado di ricevere e inseguire i riferimenti per la corrente di Plasma attraverso la rete di interconnessione tra i dispositivi, basata sul Framework di \MARTe. (argomento trattato con dettaglio nella tesi)\\
Il controllo realizzato punta a inseguire il riferimento richiesto con un errore nullo.\\
In fine il dispositivo deve essere in grado di comunicare in real-time l'attuale stato del sistema al resto della rete, per permettere una diagnostica in tempo reale dell'andamento dell'impianto.\\

\fancyhf{} %elimina header/footer vecchi


\fancyhead[R]{\rightmark} \fancyhead[L]{\leftmark}
\fancyfoot[R]{\thepage}





%---------------------
% INCLUSIONE CAPITOLI
%---------------------

%\chapter{Alcune regole fondamentali}
\label{chap:fond}

\begin{minipage}{12cm}\textit{Se lo si desidera, utilizzare questo spazio per inserire un breve riassunto di ci\`o che verr\`a detto in questo capitolo. Inserire solo i punti salienti.}
\end{minipage}

\vspace*{1cm}

\section{Come iniziare}
\label{sec:iniziare}

La tesi va scritta partendo dall'indice. Dopo aver avviato il lavoro,
lo studente deve fare uno sforzo di qualche giorno per scrivere  un
indice quanto pi\`u accurato e strutturato possibile della
relazione. L'indice va poi discusso col relatore, possibilmente prima
di incominciare a scrivere, in quanto esso influisce fortemente sul
tono da tenere nella scrittura.

L'indice pu\`o essere preparato con l'ausilio di LaTeX semplicmente
impostando le varie sezioni e affidandosi al comando
\verb1\tableofcontents1 che genera automaticamente l'indice in cima
alla tesi. Successivamente, durante la scrittura, i vari capitoli
vuoti verranno rimepiti.

ATTENZIONE a non cadere nell'errore di sottostimare il proprio lavoro
e iniziare a scrivere cose scopiazzate qua e l\`a. La relazione deve
corrispondere ad una descrizione dettagliata del lavoro fatto (\`e
questa la cosa pi\`u importante da documentare, in aiuto del relatore
e in aiuto dei tesisti che eventualmente proseguiranno il
lavoro). Tutto ci\`o che non riguarda il lavoro fatto sar\`a una parte
introduttiva scritta alla fine, anche di corsa, e di scarso
interesse. A volte gli studenti fanno l'errore di cominciare a
scrivere un lungo trattato su cose che non sono farina del loro
sacco. Quando arrivano alla vera e proria descrizione del loro lavoro,
ormai la tesi \`e gi\`a troppo lunga e sacrificano proprio quella
parte, la pi\`u importante, per mancanza di tempo e di
energie. Quindi: cominciate {\rm sempre} a scrivere dalla parte
centrale dell'indice della tesi, e poi man mano aggiungete le parti
introduttive. La tesi non viene scritta di getto dall'inizio alla
fine, come in una operazione di copiatura, ma nasce dalla sua parte
centrale, quella pi\`u importante, e poi man mano si gonfia come un
palloncino, eventualmente vedendo, durante la propria crescita, delle
revisioni dell'indice e dei cambi strutturali (quali lo swap di due
capitoli o lo spostamento di un intero capitolo in appendice)
nell'interesse della chiarezza e dell'organicit\`a del documento.


\section{Questioni di impostazione}
\label{sec:impostazione}

In questa sezione vengono commentate alcune questioni estetiche e di
forma legate alla tesi.

\subsection{La terza persona}

La tesi va scritta usando la terza persona, per quanto possibile, tranne casi veramente eccezionale. In inglese \`e piuttosto standard usare la prima persona (plurale) in testi tecnici. In italiano no.

\subsection{La lingua}

L'impostazione della lingua (italiana) \`e fondamentale perch\'e le parole vengano spezzate correttamente dal LaTeX quando deve andare a capo. Tale impostazione funziona soltanto se il LaTeX che si utilizza \`e corredato dei corrispondenti files di stile.

\subsection{La punteggiatura}
\label{sec:puntegg}

La punteggiatura va {\bf sempre} attaccata alla parola precedente e
staccata (con uno spazio) dalla parola seguente (a parte le virgolette
aperte per le quali vale la regola opposta).

\subsection{Gli accenti}
\label{sec:accenti}

In LaTeX non \`e possibile scrivere un carattere accentato semplicemente riportandolo nel codice TeX. Per farlo si deve usare un comando particolare: \verb1\`1. Ad esempio \verb1\`e1 produrr\`a \`e.\\

NOTA!!! Il carattere `` ` '' \`e diverso dal carattere `` ' ''! Col primo si ottiene \`e, col secondo \'e!\\
In Linux questo ``accento obliquo'' \`e ottenibile utilizzando la combinazione di tasti ALTGR+'. In ambiente Windows si consiglia di utilizzare la Mappa Caratteri.\\

{\bf ATTENZIONE AGLI ACCENTI}: Da un punto di vista grammaticale,
tutte le parole accentate italiano 
hanno accento grave, ovvero dall'alto verso il basso, eccezion fatta
per la lettera `` e '' che pu\`o avere sia accento acuto che grave a
seconda della parola. Pi\`u specificatamente, le `` e '' accentate sono
quasi tutte acute, a parte due parole: `` \`e '' e `` cio\`e '' (infatti
perch\'e, poich\'e, affinch\'e, etc. hanno tutte l'accento
acuto). Un'ultima osservazione va fatta per la lettera `` i ''
accentata: la si ottiene con la sequenza \verb1\`{\i}1 che d\`a il
seguente risutato: \`{\i}.

\subsection{Le virgolette}

Le virgolette aperte si ottengono con la sequenza \verb1``1 mentre
quelle chiuse si ottengono con la sequenza \verb1''1 oppure con il
carattere \verb1"1.

\subsection{Posizione delle figure}

Il LaTeX posiziona le figure automaticamente, questo significa che
esse non appariranno sempre dove ci aspettiamo di vederle. \`E dunque
fondamentale riferirsi alle figure con il comando
\verb1In figura~\ref{fig:mylabel}1 che fa riferimento ad una label
specificata dentro la figura tramite il comando
\verb1\label{fig:mylabel}1 e che consente di riferirsi alla figura con
il suo numero e senza riferimenti legati al layout del testo (tipo:
``qui sotto'' oppure `` in cima alla pagina'', oppure ``nella pagina
seguente'', etc.)

\subsection{Caption e note a pi\'e di pagina}

Le note a pi\'e di pagina 
si ottengono semplicemente digitando 
\verb1\footnote{Questo \`e il testo.}1 attaccato alla lettera
precedente (questo \`e ci\`o che 
risulta\footnote{Questo \`e il  testo.}).

Sia per le note che per le (o legende) delle figure, \`e necessario
sempre partire con la lettera maiuscola e terminare con un punto.



%\include{capitolo1}
%\include{capitolo2}
%\chapter{Conclusioni e sviluppi futuri}
\section*{Conclusioni}
In conclusione, con questo prototipo si è realizzato un controllore \textit{PID-style} a doppio Polo nell'origine, il quale si è dimostrato esser un design di controllore adatto ad un generico impianto Tokamak.\\
La realizzazione di questo prototipo però, come per ogni progetto pratico, ha dovuto confrontarsi con vari problemi:\\
\nonLinearita nel prototipo reale, problemi implementativi, limiti di attuazione e campionamento, etc...\\
La risoluzione di questi problemi ha portato a compromessi e semplificazioni volte a catturare gli aspetti principali dell'esperimento, trascurando quelli secondari.\\
Il controllo così creato, oltre a funzionare bene nella teoria, risulta robusto alle variazioni dal modello lineare, e ciò rende un simile design di controllo \textit{general-purple} per impianti Tokamak, poiché i coefficienti trovati nel corso di questa tesi permettono l'ottimizzazione per questo impianto in particolare, ma porterebbero a convergenza un qualunque impianto Tokamak avente la stessa struttura nella dinamica.\\

\section*{Sviluppi Futuri}
Anche se per i fini della tesi il lavoro termina qui, il progetto in se ancora avanza, e tra gli sviluppi futuri abbiamo:
\begin{itemize}
	\item L'intenzione di portare un controllo \textbf{Switching} nella legge di Update, variando i coefficienti del controllore \ref{eq:controllerDesign}\\
	      In tal senso il controllore implementato nel \microControllore, ottenuto dal modello nello spazio di stato in Forma Compagna di Osservatore (\cite{FormeCanoniche}) già rende il sistema pronto per questo tipo di modifica a livello di codice, è necessario tuttavia studiare e simulare per quali soglie lo switch dei parametri può incrementare le performance, prima di poter implementare una simile tecnica.
	\item Attualmente è in cantiere la realizzazione all'interno di \MARTe della libreria \cite*{EMP} per permettere l'integrazione di questo firmware con l'ecosistema \MARTe.
	\item Risulta di interesse eseguire un upgrade del \microControllore dal'\ArduinoUno a una scheda più performante, e che permetta comunicazioni, campionamenti e PWM a frequenze superiori, superando così l'attuale soglia di $ 2Khz $ nel controllo.
	\item In fine, per raggiungere performance superiori è necessario avere un motore superiore, in questo caso il Ponte-H, trovarne uno con una dinamica più lineare e che possa funzionare a frequenze maggiori di PWM permetterebbe di avere un controllo sul Primario molto più fine e potente, portando a migliorare le prestazioni in maniera sensibile e tangibile.
\end{itemize}





%\chapter*{Appendice A\\ Arduino Code}\label{ArduinoCode}
\addcontentsline{toc}{chapter}{Appendice A - Codice Arduino}

\section{Set-up Registri}

\subsubsection{Tic Timer}

\begin{lstlisting}[style=cppStyle,caption={Tic Timer},label=lst:ticTimer] 
	void periodicTask(int time) { 		// time in micro secondi
		// PWM pin Disable, motalita CTC(pt1)
		TCCR2A = (0x0 << COM2A0) | (0x0 << COM2B0) | (0x2 << WGM20);
		// CTC(pt2), Prescalere 256
		TCCR2B = (0 << WGM22) | (0x6 << CS20);                       
		// T_cklock * Twant / Prescaler = valore Registro
		OCR2A = (int)(16UL * time / 256);
		TIMSK2 = (1 << OCIE2A); // attivo solo l'interrupt di OC2A
	}
\end{lstlisting}
Questa funzione imposta il TIMER2 in modalità Fast PWM, ovvero che si resetta quando arriva al conteggio finale, e calcola il valore da mettere nel registro affinchè il conteggio sia il più vicino possibile a tempo desiderato

\subsubsection{Frequenza PWM}

\begin{lstlisting}[style=cppStyle,caption={Frequenza PWM},label=lst:pwmFreq] 
enum pwmFreq: char {
	hz30, hz120, hz490, hz4k, hz30k
};

void setMotFreq(pwmFreq freq) {
	// TCCR0B is for Timer 0
	#define myTimer TCCR0B
	switch (freq) {
		// set timer 3 divisor to  1024 for PWM frequency of    30.64 Hz
		case hz30:
			myTimer = (myTimer & B11111000) | B00000101;
		break;
		case hz120:
		// set timer 3 divisor to   256 for PWM frequency of   122.55 Hz
			myTimer = (myTimer & B11111000) | B00000100;
		break;
		case hz490:
		// set timer 3 divisor to    64 for PWM frequency of   490.20 Hz
			myTimer = (myTimer & B11111000) | B00000011;
		break;
		case hz4k:
		// set timer 3 divisor to     8 for PWM frequency of  3921.16 Hz
			myTimer = (myTimer & B11111000) | B00000010;
		break;
		case hz30k:
		// set timer 3 divisor to     1 for PWM frequency of 31372.55 Hz
			myTimer = (myTimer & B11111000) | B00000001;
		break;
		default:
			setMotFreq(hz4k);
		break;
	}
	#undef myTimer
}
\end{lstlisting}
Mediante questa funzione si modifica il valore del Prescaler per il TIMER 0, modificando la velocità di conteggio si ottiene un PWM con una periodo, e quindi frequenza, che varia.


\newpage

\section{Generatore di Segnale}

Per generare i segnali di controllo in Feed-Forward usati nel sistema, sono stati usati 2 diversi livelli di programmazione.\\
Un primo livello segnali di base, definiti su tutto $\mathbb{R}$, e usabili a piacere, e dei segnali compositi e periodici da mandare durante l'esperimento.
Tutti i segnali sono pensati per andare da -100\% <-> 100\%, è compito dell'attuazione
eliminare le deadzone e traslare il controllo al valore più opportuno

\subsection{Segnali Base}

\subsubsection{Rampa}
\begin{lstlisting}[style=cppStyle,caption={Rampa Saturata},label=lst:rampa] 
int ramp(uint64_t t, int vStart, uint64_t tStart, int vEnd, uint64_t tEnd) {
	// Saturazione
	if (t < tStart)
		return vStart;
	else if (t > tEnd)
		return vEnd;
	// Retta
	unsigned int dt = t - tStart;
	return vStart + int((vEnd - vStart) / float(tEnd - tStart) * dt);
}
\end{lstlisting}
La rampa è descritta come una retta nell'intervallo di interesse, saturata prima e dopo il tempo desiderato\\
$ RampaSat(t) =
	\left \{ \begin{array}{l c}
		v_{start} + \frac{v_{end}-v_{start}}{t_{end}-t_{start}} * (t - t_{start}) & \forall t \in [t_{start},t_{end}] \\
		v_{start}                                                                 & t<t_{start}                       \\
		v_{end}                                                                   & t>t_{start}
	\end{array}
	\right.
$

\newpage
\subsection{Segnali Composti}

\subsubsection{Onda Triangloare}
\begin{lstlisting}[style=cppStyle,caption={Onda Triangolare Periodica},label=lst:ondaTriangloare] 
int triangleSignal(uint64_t t, int msQuartPeriod) {
	static uint64_t startTic = 0;
	int dTic = t - startTic;
	int pwm = 0;
	if (dTic < ticConvert(msQuartPeriod))
		pwm = ramp(dTic, 0, 0, 100, ticConvert(msQuartPeriod));
	else if (dTic < (ticConvert(msQuartPeriod) * 3))
		pwm = ramp(dTic, 100, ticConvert(msQuartPeriod), -100, ticConvert(msQuartPeriod) * 3);
	else if (dTic < (ticConvert(msQuartPeriod) * 4))
		pwm = ramp(dTic, -100, ticConvert(msQuartPeriod) * 3, 0, ticConvert(msQuartPeriod) * 4);
	else {
		pwm = 0;
		startTic = t;
	}
	return pwm;
}
\end{lstlisting}

\subsubsection{Onda Trapezoidale}
\begin{lstlisting}[style=cppStyle,caption={Onda Trapezoidale Periodica},label=lst:ondaTrapezoidale] 
int rapidShot(uint64_t t) {
	static uint64_t startTic = 0;
	int pwmRapidShot;
	long dTic = t - startTic;
	if (dTic > t4) {
		startTic = t;
		pwmRapidShot = 0;
		dTic = t - startTic;
	}
	
	if (dTic <= t1) {
		pwmRapidShot = ramp(dTic, 0, 0, 100, t1);
	} else if (dTic <= t2) {
		pwmRapidShot = 100;
	} else if (dTic <= t3) {
		// falling ramp
		pwmRapidShot = ramp(dTic, 100, t2, 0, t3);
	} else if (dTic <= t4) {
		pwmRapidShot = 0;
	}	
	return pwmRapidShot;
}
\end{lstlisting}


\chapter*{Appendice B\\ EMP Code}\label{EMPCode}
\addcontentsline{toc}{chapter}{Appendice B - Codice EMP}

\chapter*{Appendice C\\ Matlab Post Elaborazione}\label{MatlabCode}
\addcontentsline{toc}{chapter}{Appendice C - Matlab Post Elaboration}

\chapter{Elementi Costitutivi}\label{Hardware}

\begin{minipage}{12cm}\textit{Se lo si desidera, utilizzare questo spazio per inserire un breve riassunto di ci\`o che verr\`a detto in questo capitolo. Inserire solo i punti salienti.}
\end{minipage}

\vspace*{1cm}


\section{Trasformatore - Modello di Tokamak}\label{Trasformatore}

La tesi va scritta usando la terza persona, per quanto possibile, tranne casi veramente eccezionale. In inglese \`e piuttosto standard usare la prima persona (plurale) in testi tecnici. In italiano no.

\subsection{Modellazione Fisica}
\subsection{Funzione di Trasferimento}

\newpage


\section{Sensore di Corrente}\label{CurrentSense}
Benché per gli obiettivi di controllo la lettura della corrente sul primario non è indispensabile, si è però preferito poter misurare cosa stia succedendo all'interno del sistema.\\

\subsection{Regime di funzionamento}
Per allo scopo di misurare la corrente è stato messo in serie al primario il sensore di Corrente \cite{ACS770}
Le scelte che hanno portato alla sua scelta sono: \vspace{-8mm}
\begin{center}
	\begin{tabular}[t]{|l r|}
		\hline
		Bandwidth:                             & 120 kHz                               \\
		Output rise time :                     & 4.1 $ \mu s $                         \\
		Ultralow power loss:                   & 100 $ \mu \Omega $ Resistenza Interna \\
		Single supply operation                & 4.5 to 5.5 V                          \\
		Extremely stable output offset voltage &                                       \\
		\hline
	\end{tabular}
\end{center}

\begin{figure}[h]
	\centering
	\includegraphics[width=1\textwidth]{ACS770/ACS770-Fig.png}
	\caption[Sensore di Corrente \citefield{ACS770}{series}]{Sensore di Corrente}
\end{figure}


\noindent
Delle tante varianti presenti, si è scelto di usare la \citefield{ACS770}{series}, le cui caratteristiche chiave di questa variante sono:
\begin{center}
	\begin{tabular}[t]{|l r|}
		\hline
		Primary Sampled Current: & $\pm$ 100 A     \\
		Sensitivity Sens (Typ.)  & 20(mV/A)        \\
		Current Directionality   & Bidirectional   \\
		$T_{OP}$                 & –40 to 150 (°C) \\
		\hline
	\end{tabular}
\end{center}

\noindent
Queste caratteristiche lo rendono adatto per misurare i nostri esperimenti comprendo tutti i possibili valori di corrente misurabili.\\
Essendo però il principio di funzionamento basato su un sensore a effetto Hall, ovvero una misura diretta del campo magnetico indotto, è importante tenere distante il sensore dal trasformatore che nei suoi momenti di massimo flusso, genera ovviamente un campo magnetico non indifferente.

\subsection{Funzionamento Interno}

Tra le caratteristiche chiave dell'\cite{ACS770}, troviamo il disaccoppiamento fisico tra la corrente da misurare e il circuito di misura, come è possibile vedere nel suo schema a blocchi: 

\begin{figure}[h]
	\centering
	\includegraphics[width=1\textwidth]{ACS770/ACS770-SchemaBlocchi.png}
	\caption[\citefield{ACS770}{series} Schema a Blocchi]{Schema a Blocchi}
\end{figure}

\noindent
Questa caratteristica chiave, garantisce la salvaguardia del circuito logico a valle, dai possibili eventi catastrofici a monte.\\
Esso è in oltre fornito di sensori di temperatura e sistemi di "Signal Recovery" che permettono in Hardware di compensare derive termiche e \nonLinearita, ottenendo un output assimilabile a un segnale lineare:

\begin{figure}[h]
	\centering
	\includegraphics[width=0.6\textwidth]{ACS770/ACS770-Sensibilità.png}
	\caption[\citefield{ACS770}{series} Sensibilità rispetto Temperatura]{Sensibilità}
\end{figure}

\noindent
Esso ovviamente varia in base alla temperatura, e l'errore è tanto più marcato tanto maggiore è la corrente da misurare, ma leggendo dal datasheet abbiamo che questo errore, che dipende si dalle temperature di esercizio dell'esperimento, non è mai, neanche negli esperimenti più sfortunati, superiore al $\pm2\%$.\\
Anzi, alle temperature $\approx$ 25°, si mantiene contenuto tra $\pm0.5\%$.

\begin{figure}[h]
	\centering
	\includegraphics[width=0.55\textwidth]{ACS770/ACS770-NonLin.png}
	\caption[\citefield{ACS770}{series} \nonLinearita]{Temperatura/NonLinearità}
\end{figure}

\newpage

\subsection{Connessione elettrica}

La connessione del sensore è particolarmente semplice, richiedendo esternamente solo un alimentazione stabilizzata e portando subito in uscita la misura.
\begin{figure}[h]
	\centering
	\includegraphics[width=1\textwidth]{ACS770/ACS770-Schema.png}
	\caption[\citefield{ACS770}{series} Schema di collegamento]{Schema di Collegamento}
\end{figure}

\noindent
Rispetto allo schema proposto dal datasheet, però, si è anche deciso di omettere il filtro passa-basso sulla \textbf{VIOUT}, questa scelta è stata presa per minimizzare il più possibile ritardi di misura della corrente istantanea, poiché le dinamiche del sistema sul secondario, come visto, sono di tipo derivativo, e quindi estremamente rapide.\\

\subsection{Misura}
La misura della corrente viene riportata sotto forma di tensione, la quale varia in base alla \textbf{Sensibilità} del modello in uso. 
Avendo noi il \citefield{ACS770}{series}, il datasheet riporta:
\begin{center}
	\begin{tabular}[t]{|l r|}
		\hline
		Primary Sampled Current: & $\pm$ 100 A   \\
		Sensitivity Sens (Typ.)  & 20(mV/A)      \\
		Current Directionality   & Bidirectional \\
		\hline
	\end{tabular}
\end{center}

\noindent
Ciò implica che la corrente misurata, è calcolabile come:
{\large \begin{center}
	$I_{read} = \frac{V_{Read}[V]}{V_{sense}[V/A]}$
\end{center}
}
\paragraph{Offset}
Essendo però il device ad alimentazione singola (0--5V), ma la corrente misurabile Bi-direzionale, sorge la necessità di spostare gli 0A a una tensione superiore agli 0V.\\
Il datasheet riporta che $V_{offset} = \frac{Vcc}{2}\approx$ 2.5V. Da cui deriva che la vera misura di corrente è:
{\LARGE
\begin{center}
	$I_{read} = \frac{V_{Read}-V_{offset}}{V_{sense}}\frac{[V]}{[\frac{V}{A}]}$
\end{center}
}

Al fine di poter misurare l'offset effettivo, durante il set-up viene eseguito a esperimento fermo una misura dell'offset attuale, usando la \nameref{lst:offsetCalc}.\\
Il risultato della computazione, oltre ad essere usato nel controllo è inviato al computer per la post elaborazione dei dati nei grafici.

\paragraph{Sensibilità}
Usando nell'esperimento un ADC a 10Bit con tensione di riferimento a 5V, abbiamo che la massima sensibilità del $\mu$Controllore, ovvero il suo bit meno significativo è pari a:\\

{\large\begin{center}
	$V_{step}=\frac{Vcc}{2^{10}-1} = 4,887mV$.\\
	
\end{center}}
Portando questo valore nella corrente otteniamo che la Sensibilità in corrente del $\mu$Controllore è pari a:\\

{\LARGE
\begin{center}
	$I_{step} =\frac{ V_{step}}{V_{sense}} = 244,379 mA$
\end{center}
}

Il chè rende la misura buona per osservare cosa stia accadendo, ma sicuramente non sufficientemente densa da poterla usare come parametro ingresso di controllo.

\newpage

\section{Driver di Corrente - IBT-2}\label{CurrentDriver}
Per l'attuazione del controllo di corrente nella bobina primaria del trasformatore, è stato usato il driver di corrente \cite{IBT-2} .

%todo modificare con Paint.net Per mettere trasformatore e levare linea nera
\begin{figure}[h]
	\centering
	\includegraphics[width=1\textwidth]{IBT-2/TopView.jpg}
	\caption[Driver Motori IBT-2 TopView \& PinOut]{IBT-2 TopView.}
\end{figure}

\noindent
Esso non è un comune Ponte-H integrato, per poter gestire potenze superiori è stato costruito usando 2 Half-Bridge collegati insieme mediante una opportuna logica per ricreare un normale Ponte-H.\\
Questa scheda in particola ha prestazioni interessanti per gli scopi di questa tesi, i principali sono elencati di seguito:\vspace{-8mm}
\begin{center}
	\begin{tabular}[t]{|l r|}
		\hline
		Power Input Voltage:                                     & 6 -- 27 V \\
		Peak current:                                            & 43 A      \\
		Massima Frequenza di PWM:                                & 25 kHz    \\
		Protezione Sovra Tensioni                                &           \\
		Disaccoppiamento Ingresso di Potenza/Logica di controllo &           \\
		\hline
	\end{tabular}
\end{center}
\noindent
Di particolare interesse per l'esperimento è proprio la corrente di picco gestibile:
avendo le dinamiche del sistema tempi inferiori ai 5 secondi, poter reggere correnti di picco così elevate rende
la scheda perfetta per i nostri scopi.

\subsection{Connessione di Controllo}
Il driver permette 2 modalità di funzionamento:
\begin{description}
	\item[Doppio PWM] Modalità operativa che richiede l'uso di 2 PWM\\
	      Ciascun PWM controlla uno dei 2 Half-Bridge, e per evitare di bruciare i driver devono essere controllati singolarmente, il vantaggio di questa configurazione è la possibilità di usare 2 frequenze di controllo diverse.
	\item[Singolo PWM] Modalità operativa classica di un normale Ponte-H\\
	      In questa modalità, la porta nand presente sulla scheda attua la logica di controllo opportuna per governare i 2 Half-Brige come fossero un normale Ponte-H.
\end{description}

\noindent
Per il nostro esperimento si è scelto di usare il collegamento Singolo PWM così da evitare spiacevoli sorprese e avere il PWM di controllo sempre sincronizzato.	

\newpage
\subsection{Benchmark Driver}
Il driver sulla carta à buone prestazioni, ma non sono descritte le sue \nonLinearita, per farle risaltare si sono effettuati 2 esperimenti usando differenti input di controllo:
\begin{enumerate}
	\item \nameref{lst:ondaTriangloare} \\ \vspace{-11mm}
	      \begin{figure}[h]
		      \centering
		      %Todo: Ricampionare l'onda triangolare senza filtro deadzone
		      \includegraphics[height=0.5\textheight]{IBT-2/17-triangle-Trasformatore.pdf}
		      \caption[Esperimento con Onda Triangolare]{Onda Triangolare}
	      \end{figure}\vspace{-10mm}
	      \paragraph{Dead-Zone Inferiore} L'onda triangolare si presta bene per far risaltare la problematica della Dead-Zone Inferiore, infatti in tutti gli intorni in cui il segnale passa per 0, è possibile vedere come la corrente non vari minimamente, è però possibile notare che i 2 lati non sono simmetrici tra di loro, questo è facilmente spiegabile dal fatto che il primo ha una condizione iniziale $ \neq $ 0 e di fatto stiamo ancora osservando l'esaurimento del transitorio, la soglia di Dead-Zone Inferiore è quindi calcolata vedendo il primo valore di PWM  per cui il sistema risponde a destra degli 0.      
	      
	      \newpage
	\item \nameref{lst:ondaTrapezoidale} \\
	      \begin{figure}[h]
		      \centering
		      \includegraphics[height=0.5\textheight]{IBT-2/12-rapidShot-Trasformatore.pdf}
		      \caption[Esperimento con Onda Trapezoidale]{Onda Trapezoidale}
	      \end{figure}  \vspace{-10mm}
	      \paragraph{Dead-Zone Superiore} Con questo secondo segnale, si vuole mettere in evidenza il ritardo durante la discesa della rampa, pari a circa 20ms {\small \textit{(guarda 1.1s)}}, questo ritardo è in realtà dovuto da una seconda Dead-Zone presente però ai Duty-Cycle alti del PWM.
	      \vspace{-5mm}
	      \paragraph{Disturbo 50Hz} Essendo in oltre presente un segnale costante per un pò, quando i transitori terminano risulta evidente la presenza della 50hz nel segnale della corrente proveniente dall'alimentatore, questo disturbo è però dovuto alla fonte della corrente, ovvero la 220Vac del laboratorio, il medesimo esperimento realizzato con una batteria non ha disturbi così marcati.
\end{enumerate}

\chapter{Architettura di Sistema}\label{systemDesign}

%\begin{minipage}{12cm}\textit{Se lo si desidera, utilizzare questo spazio per inserire un breve riassunto di ci\`o che verr\`a detto in questo capitolo. Inserire solo i punti salienti.}
%\end{minipage}

%\vspace*{1cm}

\section{Architettura ad alto livello}
Il progetto finale ha come obiettivo la realizzazione di un architettura di controllo per le bobine poloidali presenti nei reattori tokamak.

\begin{figure}[h]
	\centering
	\includegraphics[width=1\textwidth]{Architettura/SystemArchitetture.png}
	\caption[Schema finale dell'archiettettura di controllo]{Architettura di controllo}
\end{figure}

\noindent
Lo schema proposto realizza l'obiettivo è controllare una singola bobina, il progetto finale prevederà la ripetizione in serie del medesimo schema per il numero di bobine necessarie.\\

Dallo schema risulta evidente che tutti i componenti visti nel capitolo "\nameref{cap:1}" si relazionano con lo stesso \microControllore: l'\ArduinoUno.\\
Per riportare i dati fuori e ricevere il riferimento da inseguire nella $V_2$, è stata realizzato il \nameref{EMP}, essa è stata scritta in C++ affinché possa essere Cross-Platform.\\
Il suo compito specifico, in questo progetto, è di mettere in comunicazione l'\ArduinoUno con un nodo \MARTe installato su di una \Rasp.\\
Quest'ultimo nodo ha il compito di mettere in rete il feedback dell'esperimento, e comunicare all'\ArduinoUno eventuali cambio di riferimento. Questo ultimo tratto è realizzato mediante il protocollo \textbf{SDN}, che viaggia sopra Ethernet e dà garanzie Real-time.\\
Nella sua forma finale, il progetto prevede la riproduzione in serie di questo schema di controllo per arrivare a controllare tutte le bobine poloidali presenti in un tokamak.

\newpage

\section*{EMP - Libreria di Comunicazione Seriale\\Embedded Message Pack }\label{EMP}
\addcontentsline{toc}{section}{\protect\numberline{\thesection} EMP - Libreria di Comunicazione Seriale}

\begin{figure}[h]
	\centering
	\includegraphics[width=1\textwidth]{EMP/EMP-Logo-Background.png}
\end{figure}
\paragraph{EMP (Embedded Message Pack)} nasce con l’obiettivo di standardizzare un protocollo e creare una libreria C++ basata su classi Template, che permetta di automatizzare e standardizzare tutto il lavoro di programmazione necessario all’invio/ricezione di dei pacchetti dal formato Pre-Concordati tra 2 Device connessi Peer2Peer (Nessuna pretesa di network-ing).\\
Il raggiungimento dei suoi obiettivi, si sposa con la possibilità di supportare altre features interessanti:

\paragraph{Multiple-Package} Il protocollo di comunicazione che si è deciso di usare per EMP ha permesso di estendere il suo funzionamento e permettere il trasporto, attraverso lo stesso mezzo, di \textit{\textbf{pacchetti di tipologia e dimensione diversa}} all’interno della stessa libreria, evitando al contempo di inviare per ogni pacchetto più byte di quelli strettamente necessario. $\Rightarrow$ \textbf{Alta Efficienza}

\paragraph{Zero Tempo di negoziazione} Sempre grazie al protocollo di comunicazione, EMP è adatto ad un uso ‘Streaming’, questo perché non è necessario alcuna fase di sincronizzazione iniziale o durante la trasmissione in caso di perdita di dati, in aggiunta a ciò, EMP è in grado di scartare pacchetti errati in maniera trasparente all’utilizzatore. Tutto questo grazie al protocollo che \textbf{Auto-delimita i singoli pacchetti}. $\Rightarrow$ \textbf{Trasparenza Totale}

\paragraph{Responsabilità} Le uniche responsabilità a carico degli utilizzatori sono il riempimento dei pacchetti e la definizione degli stessi tra i 2 estremi della comunicazione.

\subsection*{Consistent Overhead Byte Stuffing (COBS)}
\addcontentsline{toc}{section}{\protect\numberline{\thesection} Protocollo - COBS}
Il protocollo di comunicazione che permette l’invio di \textbf{pacchetti diversi} e \textbf{senza fasi di negoziazione} alla base della libreria è \textbf{COBS}(\cite{COBS}).\\
Si tratta di un algoritmo per la codifica di byte, progettato per essere al tempo stesso efficiente e non ambiguo, che permette la definizione di \textit{data-pack frame} \textbf{Auto-delimiti} .

\begin{figure}[h]
	\centering
	\includegraphics[width=1\textwidth]{EMP/Cobs_encoding_with_example.png}
	\caption[Esempio di COBS]{Esempio di COBS}
\end{figure}

\subsubsection{title}


\subsection{Metodo di codifica}
\subsection{Struttura del codice}
\subsection{Benchmark}

\section{Online Sampling}
\subsection{Interconnessione Arduino $\Leftrightarrow$ Companion}
\subsection{Storage su file delle informazioni}


\section{Post Elaborazione con Matlab}
\subsection{Conversioni Dati}
\subsection{Creazione dei grafici e Filtraggio}

.\chapter{Modello teorico di Controllo}\label{cap:controlModel}

\begin{minipage}{12cm}\textit{Se lo si desidera, utilizzare questo spazio per inserire un breve riassunto di ci\`o che verr\`a detto in questo capitolo. Inserire solo i punti salienti.}
\end{minipage}

\vspace*{1cm}



\section{Controllo a Errore Nullo}

\section{Simulazione Qualitativa su Simulink}
\chapter{Sviluppo Controllo reale}\label{controlDevelop}

\begin{minipage}{12cm}\textit{Se lo si desidera, utilizzare questo spazio per inserire un breve riassunto di ci\`o che verr\`a detto in questo capitolo. Inserire solo i punti salienti.}
\end{minipage}

\vspace*{1cm}

\section{Codifica del controllore}

\section{Tuning delle costanti}

\section{Esperimenti}
\chapter{Conclusioni e sviluppi futuri}
\section*{Conclusioni}
In conclusione, con questo prototipo si è realizzato un controllore \textit{PID-style} a doppio Polo nell'origine, il quale si è dimostrato esser un design di controllore adatto ad un generico impianto Tokamak.\\
La realizzazione di questo prototipo però, come per ogni progetto pratico, ha dovuto confrontarsi con vari problemi:\\
\nonLinearita nel prototipo reale, problemi implementativi, limiti di attuazione e campionamento, etc...\\
La risoluzione di questi problemi ha portato a compromessi e semplificazioni volte a catturare gli aspetti principali dell'esperimento, trascurando quelli secondari.\\
Il controllo così creato, oltre a funzionare bene nella teoria, risulta robusto alle variazioni dal modello lineare, e ciò rende un simile design di controllo \textit{general-purple} per impianti Tokamak, poiché i coefficienti trovati nel corso di questa tesi permettono l'ottimizzazione per questo impianto in particolare, ma porterebbero a convergenza un qualunque impianto Tokamak avente la stessa struttura nella dinamica.\\

\section*{Sviluppi Futuri}
Anche se per i fini della tesi il lavoro termina qui, il progetto in se ancora avanza, e tra gli sviluppi futuri abbiamo:
\begin{itemize}
	\item L'intenzione di portare un controllo \textbf{Switching} nella legge di Update, variando i coefficienti del controllore \ref{eq:controllerDesign}\\
	      In tal senso il controllore implementato nel \microControllore, ottenuto dal modello nello spazio di stato in Forma Compagna di Osservatore (\cite{FormeCanoniche}) già rende il sistema pronto per questo tipo di modifica a livello di codice, è necessario tuttavia studiare e simulare per quali soglie lo switch dei parametri può incrementare le performance, prima di poter implementare una simile tecnica.
	\item Attualmente è in cantiere la realizzazione all'interno di \MARTe della libreria \cite*{EMP} per permettere l'integrazione di questo firmware con l'ecosistema \MARTe.
	\item Risulta di interesse eseguire un upgrade del \microControllore dal'\ArduinoUno a una scheda più performante, e che permetta comunicazioni, campionamenti e PWM a frequenze superiori, superando così l'attuale soglia di $ 2Khz $ nel controllo.
	\item In fine, per raggiungere performance superiori è necessario avere un motore superiore, in questo caso il Ponte-H, trovarne uno con una dinamica più lineare e che possa funzionare a frequenze maggiori di PWM permetterebbe di avere un controllo sul Primario molto più fine e potente, portando a migliorare le prestazioni in maniera sensibile e tangibile.
\end{itemize}





\chapter*{Appendice A\\ Arduino Code}\label{ArduinoCode}
\addcontentsline{toc}{chapter}{Appendice A - Codice Arduino}

\section{Set-up Registri}

\subsubsection{Tic Timer}

\begin{lstlisting}[style=cppStyle,caption={Tic Timer},label=lst:ticTimer] 
	void periodicTask(int time) { 		// time in micro secondi
		// PWM pin Disable, motalita CTC(pt1)
		TCCR2A = (0x0 << COM2A0) | (0x0 << COM2B0) | (0x2 << WGM20);
		// CTC(pt2), Prescalere 256
		TCCR2B = (0 << WGM22) | (0x6 << CS20);                       
		// T_cklock * Twant / Prescaler = valore Registro
		OCR2A = (int)(16UL * time / 256);
		TIMSK2 = (1 << OCIE2A); // attivo solo l'interrupt di OC2A
	}
\end{lstlisting}
Questa funzione imposta il TIMER2 in modalità Fast PWM, ovvero che si resetta quando arriva al conteggio finale, e calcola il valore da mettere nel registro affinchè il conteggio sia il più vicino possibile a tempo desiderato

\subsubsection{Frequenza PWM}

\begin{lstlisting}[style=cppStyle,caption={Frequenza PWM},label=lst:pwmFreq] 
enum pwmFreq: char {
	hz30, hz120, hz490, hz4k, hz30k
};

void setMotFreq(pwmFreq freq) {
	// TCCR0B is for Timer 0
	#define myTimer TCCR0B
	switch (freq) {
		// set timer 3 divisor to  1024 for PWM frequency of    30.64 Hz
		case hz30:
			myTimer = (myTimer & B11111000) | B00000101;
		break;
		case hz120:
		// set timer 3 divisor to   256 for PWM frequency of   122.55 Hz
			myTimer = (myTimer & B11111000) | B00000100;
		break;
		case hz490:
		// set timer 3 divisor to    64 for PWM frequency of   490.20 Hz
			myTimer = (myTimer & B11111000) | B00000011;
		break;
		case hz4k:
		// set timer 3 divisor to     8 for PWM frequency of  3921.16 Hz
			myTimer = (myTimer & B11111000) | B00000010;
		break;
		case hz30k:
		// set timer 3 divisor to     1 for PWM frequency of 31372.55 Hz
			myTimer = (myTimer & B11111000) | B00000001;
		break;
		default:
			setMotFreq(hz4k);
		break;
	}
	#undef myTimer
}
\end{lstlisting}
Mediante questa funzione si modifica il valore del Prescaler per il TIMER 0, modificando la velocità di conteggio si ottiene un PWM con una periodo, e quindi frequenza, che varia.


\newpage

\section{Generatore di Segnale}

Per generare i segnali di controllo in Feed-Forward usati nel sistema, sono stati usati 2 diversi livelli di programmazione.\\
Un primo livello segnali di base, definiti su tutto $\mathbb{R}$, e usabili a piacere, e dei segnali compositi e periodici da mandare durante l'esperimento.
Tutti i segnali sono pensati per andare da -100\% <-> 100\%, è compito dell'attuazione
eliminare le deadzone e traslare il controllo al valore più opportuno

\subsection{Segnali Base}

\subsubsection{Rampa}
\begin{lstlisting}[style=cppStyle,caption={Rampa Saturata},label=lst:rampa] 
int ramp(uint64_t t, int vStart, uint64_t tStart, int vEnd, uint64_t tEnd) {
	// Saturazione
	if (t < tStart)
		return vStart;
	else if (t > tEnd)
		return vEnd;
	// Retta
	unsigned int dt = t - tStart;
	return vStart + int((vEnd - vStart) / float(tEnd - tStart) * dt);
}
\end{lstlisting}
La rampa è descritta come una retta nell'intervallo di interesse, saturata prima e dopo il tempo desiderato\\
$ RampaSat(t) =
	\left \{ \begin{array}{l c}
		v_{start} + \frac{v_{end}-v_{start}}{t_{end}-t_{start}} * (t - t_{start}) & \forall t \in [t_{start},t_{end}] \\
		v_{start}                                                                 & t<t_{start}                       \\
		v_{end}                                                                   & t>t_{start}
	\end{array}
	\right.
$

\newpage
\subsection{Segnali Composti}

\subsubsection{Onda Triangloare}
\begin{lstlisting}[style=cppStyle,caption={Onda Triangolare Periodica},label=lst:ondaTriangloare] 
int triangleSignal(uint64_t t, int msQuartPeriod) {
	static uint64_t startTic = 0;
	int dTic = t - startTic;
	int pwm = 0;
	if (dTic < ticConvert(msQuartPeriod))
		pwm = ramp(dTic, 0, 0, 100, ticConvert(msQuartPeriod));
	else if (dTic < (ticConvert(msQuartPeriod) * 3))
		pwm = ramp(dTic, 100, ticConvert(msQuartPeriod), -100, ticConvert(msQuartPeriod) * 3);
	else if (dTic < (ticConvert(msQuartPeriod) * 4))
		pwm = ramp(dTic, -100, ticConvert(msQuartPeriod) * 3, 0, ticConvert(msQuartPeriod) * 4);
	else {
		pwm = 0;
		startTic = t;
	}
	return pwm;
}
\end{lstlisting}

\subsubsection{Onda Trapezoidale}
\begin{lstlisting}[style=cppStyle,caption={Onda Trapezoidale Periodica},label=lst:ondaTrapezoidale] 
int rapidShot(uint64_t t) {
	static uint64_t startTic = 0;
	int pwmRapidShot;
	long dTic = t - startTic;
	if (dTic > t4) {
		startTic = t;
		pwmRapidShot = 0;
		dTic = t - startTic;
	}
	
	if (dTic <= t1) {
		pwmRapidShot = ramp(dTic, 0, 0, 100, t1);
	} else if (dTic <= t2) {
		pwmRapidShot = 100;
	} else if (dTic <= t3) {
		// falling ramp
		pwmRapidShot = ramp(dTic, 100, t2, 0, t3);
	} else if (dTic <= t4) {
		pwmRapidShot = 0;
	}	
	return pwmRapidShot;
}
\end{lstlisting}


\chapter*{Appendice B\\ EMP Code}\label{EMPCode}
\addcontentsline{toc}{chapter}{Appendice B - Codice EMP}

\chapter*{Appendice C\\ Matlab Post Elaborazione}\label{MatlabCode}
\addcontentsline{toc}{chapter}{Appendice C - Matlab Post Elaboration}

% ELENCO DELLE FIGURE (OPZIONALE)
\addcontentsline{toc}{chapter}{Elenco delle figure}
\listoffigures


% BIBLIOGRAFIA
\printbibliography


%\addcontentsline{toc}{chapter}{Bibliografia}
%\begin{thebibliography}{99}
%	\bibitem{bib1}Nome Autore,
%	\emph{``Nome del libro''},
%	Nome Editore, Anno di Pubblicazione.
%	\bibitem{bib2}C. Bonivento - C. Melchiorri - R. Zanasi,
%	\emph{``Sistemi di controllo digitale''},
%	Progetto Leonardo, 1995.
%	\bibitem{IBT2} Driver-Current datasheet\\ \footnotesize\url{https://www.datsi.fi.upm.es/docencia/Informatica_Industrial/DMC/IBT2.pdf}
%\end{thebibliography}

\end{document}
