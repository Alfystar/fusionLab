\chapter{Alcune regole fondamentali}
\label{chap:fond}

\begin{minipage}{12cm}\textit{Se lo si desidera, utilizzare questo spazio per inserire un breve riassunto di ci\`o che verr\`a detto in questo capitolo. Inserire solo i punti salienti.}
\end{minipage}

\vspace*{1cm}

\section{Come iniziare}
\label{sec:iniziare}

La tesi va scritta partendo dall'indice. Dopo aver avviato il lavoro,
lo studente deve fare uno sforzo di qualche giorno per scrivere  un
indice quanto pi\`u accurato e strutturato possibile della
relazione. L'indice va poi discusso col relatore, possibilmente prima
di incominciare a scrivere, in quanto esso influisce fortemente sul
tono da tenere nella scrittura.

L'indice pu\`o essere preparato con l'ausilio di LaTeX semplicmente
impostando le varie sezioni e affidandosi al comando
\verb1\tableofcontents1 che genera automaticamente l'indice in cima
alla tesi. Successivamente, durante la scrittura, i vari capitoli
vuoti verranno rimepiti.

ATTENZIONE a non cadere nell'errore di sottostimare il proprio lavoro
e iniziare a scrivere cose scopiazzate qua e l\`a. La relazione deve
corrispondere ad una descrizione dettagliata del lavoro fatto (\`e
questa la cosa pi\`u importante da documentare, in aiuto del relatore
e in aiuto dei tesisti che eventualmente proseguiranno il
lavoro). Tutto ci\`o che non riguarda il lavoro fatto sar\`a una parte
introduttiva scritta alla fine, anche di corsa, e di scarso
interesse. A volte gli studenti fanno l'errore di cominciare a
scrivere un lungo trattato su cose che non sono farina del loro
sacco. Quando arrivano alla vera e proria descrizione del loro lavoro,
ormai la tesi \`e gi\`a troppo lunga e sacrificano proprio quella
parte, la pi\`u importante, per mancanza di tempo e di
energie. Quindi: cominciate {\rm sempre} a scrivere dalla parte
centrale dell'indice della tesi, e poi man mano aggiungete le parti
introduttive. La tesi non viene scritta di getto dall'inizio alla
fine, come in una operazione di copiatura, ma nasce dalla sua parte
centrale, quella pi\`u importante, e poi man mano si gonfia come un
palloncino, eventualmente vedendo, durante la propria crescita, delle
revisioni dell'indice e dei cambi strutturali (quali lo swap di due
capitoli o lo spostamento di un intero capitolo in appendice)
nell'interesse della chiarezza e dell'organicit\`a del documento.


\section{Questioni di impostazione}
\label{sec:impostazione}

In questa sezione vengono commentate alcune questioni estetiche e di
forma legate alla tesi.

\subsection{La terza persona}

La tesi va scritta usando la terza persona, per quanto possibile, tranne casi veramente eccezionale. In inglese \`e piuttosto standard usare la prima persona (plurale) in testi tecnici. In italiano no.

\subsection{La lingua}

L'impostazione della lingua (italiana) \`e fondamentale perch\'e le parole vengano spezzate correttamente dal LaTeX quando deve andare a capo. Tale impostazione funziona soltanto se il LaTeX che si utilizza \`e corredato dei corrispondenti files di stile.

\subsection{La punteggiatura}
\label{sec:puntegg}

La punteggiatura va {\bf sempre} attaccata alla parola precedente e
staccata (con uno spazio) dalla parola seguente (a parte le virgolette
aperte per le quali vale la regola opposta).

\subsection{Gli accenti}
\label{sec:accenti}

In LaTeX non \`e possibile scrivere un carattere accentato semplicemente riportandolo nel codice TeX. Per farlo si deve usare un comando particolare: \verb1\`1. Ad esempio \verb1\`e1 produrr\`a \`e.\\

NOTA!!! Il carattere `` ` '' \`e diverso dal carattere `` ' ''! Col primo si ottiene \`e, col secondo \'e!\\
In Linux questo ``accento obliquo'' \`e ottenibile utilizzando la combinazione di tasti ALTGR+'. In ambiente Windows si consiglia di utilizzare la Mappa Caratteri.\\

{\bf ATTENZIONE AGLI ACCENTI}: Da un punto di vista grammaticale,
tutte le parole accentate italiano 
hanno accento grave, ovvero dall'alto verso il basso, eccezion fatta
per la lettera `` e '' che pu\`o avere sia accento acuto che grave a
seconda della parola. Pi\`u specificatamente, le `` e '' accentate sono
quasi tutte acute, a parte due parole: `` \`e '' e `` cio\`e '' (infatti
perch\'e, poich\'e, affinch\'e, etc. hanno tutte l'accento
acuto). Un'ultima osservazione va fatta per la lettera `` i ''
accentata: la si ottiene con la sequenza \verb1\`{\i}1 che d\`a il
seguente risutato: \`{\i}.

\subsection{Le virgolette}

Le virgolette aperte si ottengono con la sequenza \verb1``1 mentre
quelle chiuse si ottengono con la sequenza \verb1''1 oppure con il
carattere \verb1"1.

\subsection{Posizione delle figure}

Il LaTeX posiziona le figure automaticamente, questo significa che
esse non appariranno sempre dove ci aspettiamo di vederle. \`E dunque
fondamentale riferirsi alle figure con il comando
\verb1In figura~\ref{fig:mylabel}1 che fa riferimento ad una label
specificata dentro la figura tramite il comando
\verb1\label{fig:mylabel}1 e che consente di riferirsi alla figura con
il suo numero e senza riferimenti legati al layout del testo (tipo:
``qui sotto'' oppure `` in cima alla pagina'', oppure ``nella pagina
seguente'', etc.)

\subsection{Caption e note a pi\'e di pagina}

Le note a pi\'e di pagina 
si ottengono semplicemente digitando 
\verb1\footnote{Questo \`e il testo.}1 attaccato alla lettera
precedente (questo \`e ci\`o che 
risulta\footnote{Questo \`e il  testo.}).

Sia per le note che per le (o legende) delle figure, \`e necessario
sempre partire con la lettera maiuscola e terminare con un punto.


